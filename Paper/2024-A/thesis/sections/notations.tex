\documentclass[../main.tex]{subfiles}
\graphicspath{{figures/}{../figures/}}

\begin{document}
% \todo[inline,color=green!40]{完成符号说明
%   (filename: sections/notations)}


\todo[color=green!30]{引用符号可以直接\gls{e},点击会跳转到符号表,定义在section/notations.tex 中}

% 符号表
\begin{table}[H]
    \centering·
    \renewcommand{\arrayrulewidth}{2.0pt}
    \begin{tabular}{ccc}
   \hline
    符号 & 说明 & 单位  \\ 
    \hline
    $\rho$                 & 极径                     & $m$                      \\
    $d$                   & 螺距                      &   $m$                    \\
    $\theta $                    & 极角                     & $rad$                      \\
 \((\rho _0(t),\theta _0(t))\)       & 龙头前把手第t秒在极坐标系中的位置                   & $(m,rad) $                     \\
    \(v_0\)                      & 龙头前把手前进速度                     & $\mathrm{m}\cdot s^{-1} $              \\
\((\rho _{i}(t),\theta _{i}(t))\)                   & 第\(i\)节板凳的后把手中心 第t秒在极坐标系中的位置                     & $(m,rad) $                      \\
$l_i$             &   第$i$节板凳前把手中心与后把手中心之间的距离                   & $m$                      \\
  \(v_i(t)\)                     & 第\(i\)节板凳的后把手中心在第\(t\)秒时的速度                     &  $\mathrm{m}\cdot s^{-1} $                    \\
   $w$                      & 板凳板宽                     & $m$                      \\
    $h$                   & 板凳把手中心离最近的板头距离                     & $m$                       \\
  $R$                     & 调头区域半径                     & m                      \\
    $r$                      & 后一段调头圆弧半径                      & m                       \\
   \hline
    \end{tabular}
    \end{table}

\end{document}
