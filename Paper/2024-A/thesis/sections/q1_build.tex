\documentclass[../main.tex]{subfiles}
\graphicspath{{figures/}{../figures/}}

\begin{document}
  \todo[color=green!30]{完成问题一模型的建立(sections/q1\_build)}
  \noindent \textbf{Step 1 螺线方程}
  \par 根据题意,建立以螺线中心为极点,盘入螺线所对应的极坐标方程。
  \begin{align}\label{1..1}
      \varGamma :\rho =a+b\theta 
      \end{align}
      \par 其中\(\rho\)为极径,\(\theta\)为极角,\(a\),\(b\)均为待定常数。
      \par 设龙头前把手中心的初始位置\(P_0\)的极坐标为\((\rho _0,\theta _0)\),等距螺线\(\varGamma\)过原点\(O\)和起点\(P_0\)\((\rho _0,\theta _0)\),于是将\((0,0)\)和\((\rho _0,\theta _0)\)代入\eqref{1..1}式解得
      \begin{align}\label{1.........1}
          a = 0\quad b = \frac{d_0}{2\pi} 
      \end{align}
      \par 因此,等距螺线\(\varGamma\)的极坐标方程为
      \begin{align}
      \varGamma :\rho =\frac{d_0}{2\pi}\theta\quad (0\leqslant \theta \leqslant 32\pi) \label{0.1}
      \end{align}
      \textbf{Step 2 求解龙头前把手中心在各时刻的位置}
      \par 将其转化为平面直角坐标系下的参数方程
      \begin{align}
      \begin{cases}
      x=\frac{d_0}{2\pi}\theta \cos \theta\\
      y=\frac{d_0}{2\pi}\theta \sin \theta\\
      \end{cases},(0\leqslant \theta \leqslant 32\pi) .\label{0.0}
      \end{align}
      
      \par 记龙头前把手中心在第\(t\)秒时的在第\(t\)秒时的位置为\(P_0(t)\),\(P_0(t)\)的极坐标为\((\rho _0(t),\theta _0(t))\),已知龙头前把手的行进速度为\(v_0 = 1\mathrm{m}\cdot s^{-1}\),因此从0s至ts,龙头前把手运动的路径长度为$v_0t$,极角从$\theta_0=32\pi $变为$\theta _0\left( t \right)$ ,由第一型曲线积分计算公式可得,曲线$\wideparen{P_0P_0\left( t \right) }$长度
      \begin{align}
          S=\int_{\wideparen{P_0P_0(t)}}{\mathrm{d}s}=\int_{\theta _0(t)}^{\theta _0}{\sqrt{[\rho (\theta )]^2+[\rho ' (\theta )]^2}\mathrm{d}\theta}.\label{1.1}
          \end{align}
     
      
      \par 通过求解,可以得到龙头前把手某一时间点与其极角之间的关系
      \begin{align}\label{1.........2}
          \theta _0(t)\sqrt{\theta _{0}^{2}(t)+1}+\ln (\theta _0(t)+\sqrt{(\theta _0(t))^2+1}) =\theta _0\sqrt{\theta _{0}^{2}+1}+\ln (\theta _0+\sqrt{\theta _{0}^{2}+1}) -\frac{4\pi}{d_0}v_0t .
          \end{align}
      \par 将\(t\in \{ 1,2,\cdots ,300 \}\)代入上式,可以得到龙头前把手中心在第\(t\)秒时的极角\(\theta _0(t)\),将其代入\eqref{0.1}式得到\(P_0(t)\)的极坐标\((\rho _0(t),\theta _0(t))\),再利用极坐标与直角坐标之间的转化公式
          \begin{align}
          \begin{cases}
          x_0(t)=\rho _0(t)\cos \theta _0(t)\\
          y_0(t)=\rho _0(t)\sin \theta _0(t)\\
          \end{cases},\label{1.........3}
          \end{align}
  
          \par 得到当\(t\in \{ 1,2,\cdots ,300 \}\)时,龙头前把手中心在第\(t\)秒时的的直角坐标\((x_0(t),y_0(t))\).
          \textbf{Step 3 求解板凳龙其余各节板凳的后把手中心在各时刻的位置}
      \par 设第\(i\)节板凳的后把手中心在第\(t\)秒时的位置为\(P_{i}(t)\),设\(P_{i}(t)\)的极坐标和直角坐标分别为\((\rho _{i}(t),\theta _{i}(t))\)和\((x_{i}(t),y_{i}(t))\).
      \par 设\(P_{i-1}(t)\)与\(P_{i}(t)\)之间的距离为\(| P_{i-1}(t)P_{i}(t)|(\mathrm{m}) (1\leqslant i\leqslant 223)\),第$i$节板凳前把手中心与后把手中心之间的距离为\(l_i(1\leq i\leq 223)\),据题意可知
  \begin{gather}\label{1.........4}
  | P_0(t)P_1(t)| = l_1 = 3.41 - 2\times 0.275 = 2.86(\mathrm{m});
  \\
  | P_{i-1}(t)P_{i}(t)| = l_i = 2.2 - 2\times 0.275 = 1.65(\mathrm{m}),2\leqslant i\leqslant 223.
  \end{gather}
  \par 通过Step 1,我们确定了龙头前把手中心在各个时刻的位置,由于板凳前把手与后把手之间的距离已知,且每一节板凳的后把手为下一节板凳的前把手。因此,我们可以通过第一节板凳前把手在不同时刻的位置推出其余各个把手在每个时刻的位置.
  \begin{itemize}
  \item \textbf{求解计算第1节板凳的后把手中心在第\(t\)秒时的位置\(P_1(t)\)}
  \end{itemize}
  \par 由上述内容可知,第一节板凳的前把手中心$P_0(t)$与后把手中心$P_1(t)$之间的距离为$l_{1}$,由极坐标系的两点之间距离公式可得
  \begin{align}
  l_{1}^{2}=| P_0(t)P_1(t)|^2=(\rho _1(t))^2+(\rho _0(t))^2 - 2\rho _1(t)\rho _0(t)\cos (\theta _0(t)-\theta _1(t)) .\label{2.1} 
  \end{align}
  \par 因为第一节板凳的后把手中心位于螺线\(\varGamma\)上,结合\eqref{0.1}式可得
  \begin{align}
  \rho _1(t)=\frac{d_0}{2\pi}\theta _1(t),(0\leqslant \theta _1(t)\leqslant 32\pi) .\label{2.2}
  \end{align}
  \par 通过联立\eqref{2.1}\eqref{2.2}式建立距离与极角之间的关系式
  \begin{align}
  l_{1}^{2}=\frac{d_{0}^{2}}{4\pi ^2}[(\theta _1(t))^2+(\theta _0(t))^2 - 2\theta _1(t)\theta _0(t)\cos (\theta _0(t)-\theta _1(t))] .\label{2.0}
  \end{align}
  \par 根据上式利用Python求解\(\theta _1(t)\),可能得到多个不同解.不妨设这些为不同的解为\(\alpha _{j}^{1}(t) (j = 1,2,\cdots ,m)\),据题意可知一定有\(\theta _1(t)>\theta _0(t)\),因此令
  \begin{align}\label{1.........5}
  A_1 = \{ \alpha _{j}^{1}(t) |\alpha _{j}^{1}(t) >\theta _0(t),j = 1,2,\cdots ,m \} .
  \end{align}
  \par 又因为龙头前把手与第1节板凳后把手中心的极角之差一定是最小的,所以
  \begin{align}\label{1.........6}
  \theta _1(t)=\underset{\alpha _j(t)\in A_1}{\min}\,\left[ \alpha _{j}^{1}(t)-\theta _0\left( t \right) \right] +\theta _0\left( t \right) .
  \end{align}
  \par 将通过上述限制求得的解\(\theta _1(t)\)代入\eqref{2.2}式得到此时\(P_1(t)\)的极坐标\((\rho _1(t),\theta _1(t))\).
  \par 令\(t\)依次取\(1\),\(2\),\(\cdots\),\(300\),反复进行上述操作得到当\(t\in \{ 1,2,\cdots ,300 \}\)时,\(P_1(t)\)的极坐标\((\rho _1(t),\theta _1(t))\),进而通过极坐标与直角坐标转化公式:
  \begin{align}
  \begin{cases}
  x_1(t)=\rho _1(t)\cos \theta _1(t)\\
  y_1(t)=\rho _1(t)\sin \theta _1(t)\\
  \end{cases}, \label{1.........7}   
  \end{align}
  \par 得到第1节板凳的后把手中心在第\(t\)秒时的位置直角坐标\((x_1(t),y_1(t))\).
  \begin{itemize}
      \item \textbf{求解计算第\(i(2\leqslant i\leqslant 223)\)节板凳的后把手中心在第\(t\)秒时的位置\(P_{i}(t)\)}
      \end{itemize} 
      \par 通过Step 1,我们确定了龙头前把手中心在各个时刻的位置,由于板凳前把手与后把手之间的距离已知,且每一节板凳的后把手为下一节板凳的前把手。因此,我们可以通过第一节板凳前把手在不同时刻的位置推出其余各个把手在每个时刻的位置.
      \begin{itemize}
      \item \textbf{求解计算第1节板凳的后把手中心在第\(t\)秒时的位置\(P_1(t)\)}
      \end{itemize}
  \par 此问与上疑问求解第一节板凳后把手中心位置思路一致,变量为前把手与后把手之间的距离。因此,我们可以得到公式:
  \begin{gather}
      l_{i}^{2}=| P_{i-1}(t)P_{i}(t)|^2=(\rho _{i}(t))^2+(\rho _{i-1}(t))^2 - 2\rho _{i}(t)\rho _{i-1}(t)\cos (\theta _{i-1}(t)-\theta _{i}(t)) .\label{2.2.1}
      \\
      \rho _{i}(t)=\frac{d_0}{2\pi}\theta _{i}(t),(0\leqslant \theta _{i}(t)\leqslant 32\pi) .\label{2.2.2}
      \end{gather}
      \par 通过联立\eqref{2.2.1}\eqref{2.2.2}式建立距离与极角之间的关系式
      \begin{align}
          l_{i}^{2}=\frac{d_{0}^{2}}{4\pi ^2}[(\theta _{i}(t))^2+(\theta _{i-1}(t))^2 - 2\theta _{i}(t)\theta _{i-1}(t)\cos (\theta _{i-1}(t)-\theta _{i}(t))] .\label{2.0.0}
          \end{align}
      \par 根据上式利用Python求解\(\theta _{i}(t)\),可能得到多个不同解.不妨设这些为不同的解为\(\alpha _{j}^{i}(t) (j = 1,2,\cdots ,m)\),注意到一定有\(\theta _{i}(t)>\theta _{i-1}(t)\),因此令
      \begin{align}\label{1.........9}
      A_i = \{ \alpha _{j}^{i}(t) |\alpha _{j}^{i}(t) >\theta _{i-1}(t),j = 1,2,\cdots ,m \},
      \end{align}
      \par 又因为第\(i - 1\)个把手与第$i$个把手的极角之差一定是最小的,所以
      \begin{align}\label{1.........10}
          \theta _i(t)=\underset{\alpha _{j}^{i}(t)\in A_i}{\min}\left[ \alpha _{j}^{i}\left( t \right) -\theta _{i-1}\left( t \right) \right] +\theta _{i-1}\left( t \right) .  
          \end{align}
          \par 将上述求得的\(\theta _i(t)\)代入\eqref{2.2.2}式就能得到此时\(P_{i}(t)\)的极坐标\((\rho _{i}(t),\theta _{i}(t))\).令\(t\)依次取\(1\),\(2\),\(\cdots\),\(300\),反复进行上述操作就能得到,当\(t\in \{ 1,2,\cdots ,300 \}\)时,\(P_{i}(t)\)的极坐标\((\rho _{i}(t),\theta _{i}(t))\).再利用
          \begin{align}
          \begin{cases}
          x_{i}(t)=\rho _{i}(t)\cos \theta _{i}(t)\\
          y_{i}(t)=\rho _{i}(t)\sin \theta _{i}(t)\\
          \end{cases},\label{1.........11} 
          \end{align}
          \par 得到第$i$节板凳的后把手中心在第\(t\)秒时的位置直角坐标\((x_i(t),y_i(t))\).
      \par 综上所述,通过Step 2中求得的龙头前把手在不同时刻的位置,我们可以根据Step 3中求得的公式不断迭代计算,得到每秒板凳龙各把手中心的直角坐标。
   \\\textbf{Step 4 求解板凳龙各节板凳把手中心在各时刻的速度}
  \par 由题意可知龙头前把手中心在各时刻的运动速度,即\(v_0(t) = 1\mathrm{m}\cdot s^{-1}\)。由于每节板凳的前把手中心为上一节板凳的后把手中心,所以求解板凳龙其余板凳中心在各个时刻的速度即求解每节板凳的后把手中心在各个时刻的速度。对于板凳龙其余各节板凳后把手在各个时刻的速度,我们可以通过计算每个把手对应的弧长关于时间的微分获得。
  \par 记第\(i(1\leqslant i\leqslant 223)\)节板凳的后把手中心在第\(t\)秒时的速度为\(v_i(t)\).
  设第\(i(1\leqslant i\leqslant 223)\)节板凳的后把手中心在充分短的时间\(\mathrm{d}t\)内经过的路程微分为\(\mathrm{d}s_i\),因为各节板凳把手中心始终在螺线\(\varGamma\)上,所以各节板凳把手中心经过的路程微分\(\mathrm{d}s_i\)就是螺线\(\varGamma\)的弧微分,因此利用\eqref{0.1}式及弧微分的计算公式可得
  \begin{align}\label{1.........12}
  \mathrm{d}s_i=\sqrt{[\rho (\theta _i)]^2+[\rho \prime (\theta _i)]^2}\mathrm{d}\theta  _i=\frac{d_0}{2\pi}\sqrt{{\theta _i}^2+1}\mathrm{d}\theta _i,i\in \{ 1,2,\cdots ,223 \}  
  \end{align}
  故当\(t\in \{ 1,2,\cdots ,300 \}\)时,由瞬时速度的定义可得
  \begin{align}
  v_i(t) =\frac{\mathrm{d}s_i}{\mathrm{d}t}=\frac{d_0}{2\pi}\frac{\sqrt{{\theta _i}^2 + 1}\mathrm{d}\theta _i}{\mathrm{d}t}, i\in \{1, 2, \cdots, 223\}. \label{3.1}
  \end{align}
  \par 在公式中,我们可以看到极角对于各板凳后把手速度的重要影响。因此,我们需要求取各节板凳后把手在$t$时刻下的极角递推出上一节板凳后把手在$t$时刻下的极角的递推公式,从而建立相邻两节板凳后把手中心在$t$时刻下速度的递推公式,得到任一把手在任一时刻下的速度。
  \par 当\(i\in \{ 1,2,\cdots ,223 \}\)时,将\eqref{2.0}\eqref{2.0.0}式两边同时对\(t\)求导:
  \begin{align}
      \theta_{1}'(t)&=\frac{\theta_{0}'(t)[\theta_{0}(t)-\theta_{1}(t)\cos(\theta_{0}(t)-\theta_{1}(t))+\theta_{1}(t)\theta_{0}(t)\sin(\theta_{0}(t)-\theta_{1}(t))]}{\theta_{1}(t)-\theta_{0}(t)\cos(\theta_{0}(t)-\theta_{1}(t))-\theta_{1}(t)\theta_{0}(t)\sin(\theta_{0}(t)-\theta_{1}(t))} \label{3.1.2.3}
  \\
  \theta_{i}'(t)&=\frac{\theta_{i - 1}'(t)[\theta_{i - 1}(t)-\theta_{i}(t)\cos(\theta_{i - 1}(t)-\theta_{i}(t))+\theta_{i}(t)\theta_{i - 1}(t)\sin(\theta_{i - 1}(t)-\theta_{i}(t))]}{\theta_{i}(t)-\theta_{i - 1}(t)\cos(\theta_{i - 1}(t)-\theta_{i}(t))-\theta_{i}(t)\theta_{i - 1}(t)\sin(\theta_{i - 1}(t)-\theta_{i}(t))}\label{3.1.1.3}
  \end{align}
  \par 为使等式简洁,设\(\theta _i = \theta _i(t)\),\(\theta _{i - 1} = \theta _{i - 1}(t)\) ,两式相除并化简:
      \begin{align}
  \frac{\mathrm{d}\theta _i}{\mathrm{d}t}=\frac{-\theta _{i - 1}+\theta _i\cos(\theta _{i }-\theta _{i-1})+\theta _i\theta _{i - 1}\sin(\theta _{i }-\theta _{i-1})}{\theta _i+\theta _i\theta _{i - 1}\sin(\theta _i -\theta _{i-1})-\theta _{i - 1}\cos(\theta _i -\theta _{i-1})}\cdot \frac{\mathrm{d}\theta _{i - 1}}{\mathrm{d}t}.\label{3.0}
  \end{align}
  \par 联立\eqref{3.1}\eqref{3.0}式得到
  \begin{align}
  |v_i(t)| &= \left|\frac{d_0}{2\pi}\frac{\sqrt{{\theta _i}^2 + 1}\mathrm{d}\theta _i}{\mathrm{d}t}\right| 
  = \frac{|-\theta _{i - 1}+\theta _i\cos(\theta _{i }-\theta _{i-1})+\theta _i\theta _{i - 1}\sin(\theta _{i }-\theta _{i-1})|}{|\theta _i+\theta _i\theta _{i - 1}\sin(\theta _i -\theta _{i-1})-\theta _{i - 1}\cos(\theta _i -\theta _{i-1})|}\sqrt{\frac{1 + {\theta _i}^2}{1 + {\theta _{i - 1}}^2}}\left|\frac{\mathrm{d}\theta _{i - 1}}{\mathrm{d}t}\right| \label{1.........15}\\
  &= \frac{|-\theta _{i - 1}+\theta _i\cos(\theta _{i }-\theta _{i-1})+\theta _i\theta _{i - 1}\sin(\theta _{i }-\theta _{i-1})|}{|\theta _i+\theta _i\theta _{i - 1}\sin(\theta _i -\theta _{i-1})-\theta _{i - 1}\cos(\theta _i -\theta _{i-1})|}\sqrt{\frac{1 + {\theta _i}^2}{1 + {\theta _{i - 1}}^2}}|v_{i - 1}(t)|, i\in \{1, 2, \cdots, 223\}.\label{1.........16}
  \end{align}
  \par 因为\(v_i(t) (1\leqslant i\leqslant 223)\)均大于\(0\),所以上式可化为
  \begin{align}\label{1.........17}
  v_i(t) = \sqrt{\frac{1 + {\theta _i}^2}{1 + {\theta _{i - 1}}^2}}\frac{|-\theta _{i - 1}+\theta _i\cos(\theta _{i }-\theta _{i-1})+\theta _i\theta _{i - 1}\sin(\theta _{i }-\theta _{i-1})|}{|\theta _i+\theta _i\theta _{i - 1}\sin(\theta _i -\theta _{i-1})-\theta _{i - 1}\cos(\theta _i -\theta _{i-1})|}v_{i - 1}(t), i\in \{1, 2, \cdots, 223\},
  \end{align}
  \par 据Step 1,Step 2,我们已知第\(i(1\leqslant i\leqslant 223)\)节板凳的后把手中心第\(t\)秒时的位置\(P_{i}(t)\)的极坐标\((\rho _{i}(t),\theta _{i}(t))\),第一节板凳前把手中心在各时刻的速度,因此,令\(i\)依次取\(1, 2, \cdots, 223\),再利用Python进行迭代计算,就能得到板凳龙的第\(i(1\leqslant i\leqslant 223)\)节板凳的后把手中心在第\(t\)秒时的速度\(v_i(t)\).
  再令\(t\)依次取\(1, 2, \cdots, 300\),反复进行上述操作,就能得到当\(t\in \{1, 2, \cdots, 300\}\)时,板凳龙的第\(i(1\leqslant i\leqslant 223)\)节板凳的后把手中心每秒的速度,从而得到板凳龙任一把手的速度。
  
  

\end{document}