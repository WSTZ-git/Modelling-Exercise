\documentclass[../main.tex]{subfiles}
\graphicspath{{figures/}{../figures/}}

\begin{document}
\todo[color=green!30]{完成问题重述(sections/question\_review)}
\subsection{问题背景}
“板凳龙”是浙闽地区的一项富有特色的传统地方民俗文化活动,其主要形式是将数十至上百条板凳首尾相连,构成一条蜿蜒曲折的板凳龙。在表演过程中,龙头引领,龙身与龙尾随之盘旋,整体形成圆盘状。这一活动的观赏性高度依赖于舞龙队的操控能力,具体体现在舞龙队能够流畅完成盘入和盘出动作的前提下,盘龙所占的面积越小、行进速度越快,则观赏效果越佳。  

\subsection{问题一}

舞龙队沿螺距为55cm的等距螺线顺时针从螺线第16圈A点处盘入,把手中心位于螺线上,龙头前把手的行进速度始终保持1m/s。求从初始时刻到300s为止,每秒龙头、龙身和龙尾各前把手及龙尾后把手中心的位置和速度。并列出0s、60s、120s、180s、240s、300s时,龙头前把手、龙头后面第151、101、151、201节龙身前把手和龙尾后把手的位置和速度。

\subsection{问题二}
确定舞龙队盘入的终止时刻(板凳之间不发生碰撞),此时舞龙队的位置和速度。列出此时龙头前把手、龙头后面第 1、51、101、151、201 条龙身前把手和龙尾后把手的位置和速度。

\subsection{问题三}
舞龙队盘出的调头空间是以螺线中心为圆心、直径为 9 m 的圆形区域,求最小螺距,使得龙头前把手能够沿着相应的螺线盘入到调头空间的边界。
\subsection{问题四}
盘入螺线的螺距为1.7m,盘出螺线与盘入螺线关于螺线中心呈中心对称,龙队在问题3设定的调头空间内完成调头,调头路径是由两段圆弧相切连接而成的S形曲线,前一段圆弧的半径是后一段的2倍,它与盘入、盘出螺线均相切。能否调整圆弧,仍保持各部分相切,使得调头曲线变短?
\par 龙头前把手的行进速度始终保持不变。以调头开始时间为零时刻,给出从﹣100s开始到100为龙队的位置和速度。
列出51、101、151、201节龙身前把手和龙尾后把手的位置和速度。

\subsection{问题五}
舞龙队沿问题4设定的路径行进,龙头行进速度保持不变,请确定龙头的最大行进速度,使得舞龙队各把手的速度均不超过2m/s。

\end{document}