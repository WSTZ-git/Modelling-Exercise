\documentclass[../main.tex]{subfiles}
\graphicspath{{figures/}{../figures/}}

\begin{document}
% \todo[inline,color=green!40]{完成符号说明
%   (filename: sections/notations)}


\todo[color=green!30]{引用符号可以直接\gls{e},点击会跳转到符号表,定义在section/notations.tex 中}


\begin{table}[H]
    \centering
    \renewcommand{\arrayrulewidth}{2.0pt}
    \begin{tabular}{p{4cm}p{5cm}p{4cm}}
    \hline
    符号 & 说明 & 单位  \\ 
    \hline
    $P_i$         & 定日镜中心坐标         & $-$                      \\
    $\theta _l$   & 径向角                & $mrad$                   \\
    $\beta _l$    & 方向角                & $rad$                    \\
    $\alpha_{\text{s}}$      & 太阳高度角       & $rad$               \\
    $ \gamma _{\text{s}}$    & 太阳方位角       & $rad $              \\
    $\vec{e}_j$              & 入射光线单位向量  & $-$                 \\
    $vec{e}_{\text{k}}$      & 反射光线单位向量  & $-$                 \\
    $\vec{e}$                & 定日镜法向量     & $-$                  \\
    $\alpha_{P_{i}}$         & 定日镜俯仰角     & $rad$                \\
    $\gamma_{P_{i}}$         & 定日镜方位角     & $rad$                \\
    $\eta$                   & 定日镜光学效率   & $-$                   \\
    $\eta_{\text{sb}}$       & 阴影遮挡效率     &$-$                    \\
    $\eta_{\text{cos}}$      & 余弦效率        &$-$                    \\
    $\eta_{\text{at}}$       & 大气透射率       &$-$                    \\
    $\eta_{\text{trunc}}$    & 集热器截断效率   &$-$                    \\
    $\eta_{\text{ref}}$      & 镜面反射率       &$-$                    \\   
    $E_{\text{field}}$       & 定日镜场的输出热功率     &$\text{kW/m}^2$                    \\
    $M$                      & 定日镜总数     &$-$                    \\
    $N$                      & 总光线数量      &$-$                    \\
    \(lx\)                   & 定日镜的高度    &$m$                    \\
    \(ly\)                   & 定日镜的宽度    &$m$                    \\
    $h$         & 安装高度    &$m$                    \\
    \hline
    \end{tabular}
    \end{table}

\end{document}
