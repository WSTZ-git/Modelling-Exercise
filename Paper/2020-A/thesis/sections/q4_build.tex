\documentclass[../main.tex]{subfiles}
\graphicspath{{figures/}{../figures/}}

\begin{document}
\todo[color=green!40]{完成问题四模型的建立(sections/q4\_build)}
\par  由问题三设定的条件可知,$t_1$为温度第一次超过$217^{\circ}C$的时刻即$\min \left\{ t|T\ge 217 \right\} $,$t_2$为达到峰值温度的时刻即$max T(0,t)$,以$\varDelta t$为步长,令
\begin{align}\label{9.1}
    n=\frac{\varDelta t}{t_2-t_1}
\end{align}
\par 故而,对于$t_1\sim t_2$内的时间点$t$可以以下形式表示
\begin{align}\label{9.2}
t=t_1+i\varDelta t\ \left( i=1,2,3\cdot \cdot \cdot \cdot \cdot n \right) 
\end{align}
\par 其关于$t_2$对应的时间点为$2t_2-\left( t_1+i\varDelta t \right) \left( i=1,2,3\cdot \cdot \cdot \cdot \cdot n \right)$ .为了衡量炉温曲线关于峰值温度的对称性,我们引入一个评价指标$ M$,其表达式为以下方程
\begin{align}
    M = \sqrt{\frac{1}{n} \sum_{i = 0}^{n} \left( F(0, t_1 + i\Delta t) - F(0, 2t_2 - (t_1 + i\Delta t)) \right)^2}
\end{align}
\par 由题意可知以峰值温度为中心线的两侧超过 \(217^{\circ}C\) 的炉温曲线应尽量对称, $M $值越小,说明炉温曲线在峰值温度两侧超过$ 217^{\circ}C$ 的部分越对称。因此,我们的目标函数是使 $M $最小化,即
\begin{align}
    \min\text{\ }M= \sqrt{\frac{1}{n} \sum_{i = 0}^{n} \left( F(0, t_1 + i\Delta t) - F(0, 2t_2 - (t_1 + i\Delta t)) \right)^2}
\end{align}
\par 题目要求满足所求得的炉温曲线满足制程条件以及第三问的要求,因此建立模型为
\begin{align}
    \begin{array}{c}
        \underset{T_1,T_2,T_3,T_4,v}{min}\,\,S\\
        \min E\\
        \left\{ \begin{array}{l}
        S=\int_{t_1}^{t_2}{\left( T\left( 0,t \right) -217 \right)}dt\\
        t_1=\min \left\{ t|T\ge 217 \right\}\\
        t_2=maxT\left( 0,t \right)\\
        165\leq T_1\leq 185\\
        185\leq T_2\leq 205\\
        225\leq T_3\leq 245\\
        245\leq T_4\leq 265\\
        0\leq \left| \frac{\text{d}T\left( 0,t \right)}{\text{d}t} \right|\leq 3\\
        60\leq \Delta t\{150^{\circ}C\leq T\left( 0,t \right) \leq 190^{\circ}C\}\leq 120\\
        40\leq \Delta t\{T\left( 0,t \right) >217^{\circ}C\}\leq 90\\
        240\leq \max T\left( 0,t \right) \leq 250\\
        65\leq v\leq 100\\
    \end{array} \right.\\
    \end{array}
\end{align}
\par 我们可以根据上述模型求出最优炉温曲线,各温区的设定温度以及传送带的过炉速度。
\end{document}