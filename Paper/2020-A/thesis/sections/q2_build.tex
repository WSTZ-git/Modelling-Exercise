\documentclass[../main.tex]{subfiles}
\graphicspath{{figures/}{../figures/}}

\begin{document}
  \todo[color=green!30]{完成问题二模型的建立(sections/q2\_build)}
 
\noindent \textbf{step 1 目标函数}
 \par 根据题目要求,我们需要找到传送带的最大过炉速度,因此我们令速度最大作为目标函数
 \begin{align}\label{6.1}
  \max\text{\ }v
 \end{align}

 \noindent \textbf{step 2 约束条件}
    \par  由题目给定的制程条件,可以得知:
  \begin{itemize}
    \item 温度曲线斜率的绝对值最高为$3°\text{C/}s$,最低为$0°\text{C/}s$,即
    \begin{align}\label{6.2}
      0\leq\left|\frac{\mathrm{d}T(0,t)}{\mathrm{d}t}\right|\leq 3
    \end{align}
    \item 温度上升过程中,在$150^{\circ}C\sim190^{\circ}C$的时间最高值为$120s$,最低值为$60s$,即
   \begin{align}\label{6.3}
    60\leq\Delta t\{ 150^{\circ}C\leq T\leq190^{\circ}C\}\leq120
   \end{align}
   \item 温度大于$217^{\circ}C$的时间最大为$120s$,最小为$60s$,即
   \begin{align}\label{6.5}
    40\leq\Delta t\{T > 217^{\circ}C\}\leq90 
   \end{align}
    \item 峰值温度最高值为$250^{\circ}C$,最低值为$240^{\circ}C$,即
    \begin{align}\label{6.4}
      240\leq\max T(0,t)\leq250
    \end{align}
    \item 传送带的过炉速度调节范围为$65\sim100 cm/min$,即
    \begin{align}\label{6.6}
      65\leq v\leq100
    \end{align}
  \end{itemize}
  \par 综上所述,约束条件方程组为
  \begin{align}\label{6.7}
    \left\{\begin{array}{l}
      0\leq\left|\frac{\mathrm{d}T(0,t)}{\mathrm{d}t}\right|\leq 3
    \\ 
      60\leq\Delta t\{ 150^{\circ}C\leq T(0,t)\leq190^{\circ}C\}\leq120
    \\
      40\leq\Delta t\{T(0,t) > 217^{\circ}C\}\leq90 
    \\
      240\leq\max T(0,t)\leq250
    \\
      65\leq v\leq100
    \end{array} \right.      
  \end{align}
  \noindent \textbf{step 3 模型建立}
  \par 根据以上讨论的目标函数与约束条件,可以建立模型为
  \begin{align}
\begin{array}{c}
	\max\text{\ }v\\
	\left\{ \begin{array}{l}
	0\leq \left| \frac{\text{d}T\left( 0,t \right)}{\text{d}t} \right|\leq 3\\
	60\leq \Delta t\{150^{\circ}C\leq T\left( 0,t \right) \leq 190^{\circ}C\}\leq 120\\
	40\leq \Delta t\{T\left( 0,t \right) >217^{\circ}C\}\leq 90\\
	240\leq \max T\left( 0,t \right) \leq 250\\
	65\leq v\leq 100\\
	h\theta \left( \delta ,t \right) =-\lambda \left. \frac{\partial \theta \left( y,t \right)}{\partial y} \right|_{y=\delta}\\
	\left. \frac{\partial \theta \left( y,t \right)}{\partial y} \right|_{y=0}=0\\
	T\left( y,0 \right) =25^{\circ}\text{C\,\,}\left( 0\leq y\leq \delta \right)\\
	\frac{\partial \theta \left( y,t \right)}{\partial t}=a\frac{\partial ^2\theta \left( y,t \right)}{\partial y^2}\,\,\left( 0<y<\delta ,t>0 \right)\\
\end{array} \right.\\
\end{array}
  \end{align}
  \par 我们可以根据上述模型确定传送带过炉速度的范围,取范围的最大值,即传送带的最大过炉速度。
 
\end{document}