\documentclass[../main.tex]{subfiles}
\graphicspath{{figures/}{../figures/}}

\begin{document}
\begin{abstract}
% % \todo[color=green!30]{完成摘要 (sections/abstract)}

\textbf{针对问题一}:以假目标为原点建立直角坐标系,通过匀速直线运动公式计算导弹$M1$的实时位置,结合圆柱坐标方程确定真目标侧面离散点;基于无人机$FY1$的投放参数与烟幕弹运动规律,推导烟幕云团的球面方程;通过联立导弹到真目标连线方程与云团球面方程,结合判别式$\Delta$判断相交关系,离散化时间轴遍历验证有效遮挡时刻。最后得到烟幕干扰弹对$M1$的有效遮蔽时间为$\textbf{1.391982s}$。

\textbf{针对问题二}:以$FY1$的飞行方向、烟幕弹投放时间$t_1$、起爆时间$t_2$为决策变量,以有效遮蔽时间最大化为目标函数,建立单目标优化模型;采用\textbf{模拟退火算法},规避局部最优,离散化时间轴验证遮蔽状态。最后得到当$FY1$飞行方向与$x$轴夹角$\mathbf{3.13rad}$、速度$\mathbf{109.78m/s}$,投放点$\mathbf{(17722.061437, 0.903555, 1800.00000)}$、起爆点$(\mathbf{17335.661803}$, $\mathbf{5.383153}$, $\mathbf{1739.287040})$时,最长遮蔽时间为$\mathbf{4.602140s}$。

\textbf{针对问题三}:在问题二基础上,增加烟幕弹投放间隔约束,以$FY1$飞行参数及3枚烟幕弹的投放、起爆时间为决策变量,仍以总遮蔽时间最大化为目标,采用\textbf{模拟退火算法}迭代优化,遍历离散化的真目标点与时间点验证遮蔽效果。最后得到当$FY1$飞行方向$\mathbf{179.66}$度、速度$\mathbf{140}$m/s时,3枚烟幕弹投放点$x$坐标分别为$\mathbf{17653.00}m$、$\mathbf{17294.61}m$、$\mathbf{17112.61}m$,对应有效干扰时长依次为$\mathbf{4.1}s$、$\textbf{2.7}s$、$\mathbf{1.7}s$,最大有效遮蔽时间为$\mathbf{6.719452}$s。

\textbf{针对问题四}:考虑$FY1$、$FY2$、$FY3$三架无人机,以每架无人机的飞行方向、速度及单枚烟幕弹的投放、起爆时间为决策变量,目标为总遮蔽时间最大化;采用\textbf{差分进化算法},同步验证三枚烟幕云团的时空遮蔽覆盖。最后得到$FY1$方向$\mathbf{4.86}$度、速度$\mathbf{96.37}$m/s,$FY2$方向$\mathbf{246.86}$度、速度$\mathbf{87.25}$m/s,$FY3$方向$\mathbf{100.85}$度、速度$\mathbf{93.93}$m/s,其烟幕弹有效干扰时长分别为$\mathbf{4.60}s$、$\mathbf{2.95}s$、$\mathbf{3.10}s$;最大有效遮蔽时间为$\mathbf{11.50}s$。并且根据灵敏度分析可知,\textbf{导弹速度与有效遮蔽时间呈负相关,速度越快遮蔽时间越短}。

\textbf{针对问题五}:针对5架无人机(每架至多3枚烟幕弹)与3枚敌方导弹,先独立优化单无人机对三枚导弹的投放策略,再合并去重时间计算总遮蔽时长;以无人机飞行参数、烟幕弹投放/起爆时间为变量,考虑投放间隔与云团寿命约束,采用\textbf{差分进化算法}分阶段寻优,验证多弹对多导弹的协同遮蔽。最后得到$FY1$方向$\mathbf{7.45}$度、速度$\mathbf{72}$m/s,$FY2$方向$\mathbf{295.07}$度、速度$\mathbf{127.02}$m/s等5架无人机的烟幕弹对$M1$、$M2$、$M3$的有效遮蔽时长存在差异,其中$FY2$对$M1$贡献$\mathbf{4.12}$s、对$M2$贡献$\mathbf{3.38}$s,$FY3$对$M1$贡献$\mathbf{1.75}$s、对$M2$贡献$\mathbf{2.77}$s;最大有效遮蔽时间为$\mathbf{22.97}$s。

\keywords{\textbf{单目标优化} \qquad
\textbf{遍历算法} \qquad \textbf{模拟退火} \qquad \textbf{差分净化算法} 
}
\end{abstract}
\end{document}
