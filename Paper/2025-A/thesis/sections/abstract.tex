\documentclass[../main.tex]{subfiles}
\graphicspath{{figures/}{../figures/}}

\begin{document}
\begin{abstract}
% % \todo[color=green!30]{完成摘要 (sections/abstract)}

\par \textbf{针对问题一,}首先,我们已知导弹 \( M1 \) 和假目标在题目给定坐标系下的初始位置,以及导弹直指假目标的飞行方向和飞行速度,因此,我们可以得到导弹 \( M1 \) 在 \( t \) 时刻下的坐标。通过题目给定条件,我们可以得到真目标所在的圆柱侧面各点的位置,由此可知在\( t \)时刻导弹\( M_1 \)与真目标上任一点的连线方程;
其次,我们已知无人机 \( FY1 \) 的初始位置,无人机匀速直线运动至假目标的速度,进而得到烟幕干扰弹投放时的位置,已知烟幕干扰弹在重力作用下运动,从而得到烟幕干扰弹爆炸时的位置,也就是云团中心的初始位置,已知云团匀速下沉的速度,进而可以知道得到在 \( t \) 时刻云团球体球心的位置,进而得到云团球体球面上各点的位置;
最后,将\( t \)时刻导弹\( M1 \)与真目标上任一点的连线方程与云团球体在 \( t \) 时刻下的球面方程进行联立,可以判断在 \( t \) 时刻云团是否对 \( M1 \) 进行遮挡,从而得到烟幕干扰弹对 \( M1 \) 的有效遮挡时长。
\par \textbf{针对问题二,}首先;然后;最后。
\\
\par \textbf{针对问题三,}首先;然后;最后。
\keywords{关键词1 \qquad
关键词2 
}
\end{abstract}
\end{document}
