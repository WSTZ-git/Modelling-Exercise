\documentclass[../main.tex]{subfiles}
\graphicspath{{figures/}{../figures/}}

\begin{document}
% \todo[color=green!40]{完成问题四模型的建立(sections/q4\_build)}
\noindent \textbf{Step 1 目标函数}
\par 在本问中,我们的目标是通过设计3架无人机$FY1$,$FY2$,$FY3$与其分别投放的1枚烟幕干扰弹相关参数,使得这3架无人机分别释放的3枚烟幕干扰弹对导弹$M1$的有效遮蔽总时间尽可能长,因此目标函数为:
\begin{align}\label{19.1}
  \underset{\alpha_j ,v_{FYj},t_{FYj,j1},t_{FYj,j2}}{\max}\Delta t_{31}
\end{align}
\noindent \textbf{Step 2 决策变量}

\begin{itemize}
\item \textbf{无人机$FYj$的方向}:设$\alpha_j$为无人机$FYj$与$x$轴正方向的夹角,逆时针为正,范围为$[ 0,2\pi ) $,从而确定无人机$FYj$的飞行方向$(j=1,2,3)$。
\item  \textbf{无人机$FYj$的飞行速度} :设无人机$FYj$的飞行速度为 \( v_{FYj} \)。

\item \textbf{烟幕干扰弹投放点}:设无人机$FYj$在受领任务 \( t_{FYj,11} \) s 后投放1枚烟幕干扰弹。无人机受领任务后,保持等高度匀速直线运动。由问题一求得的\eqref{1.8}可知 \( t \) 时刻无人机 \( FYj \) 的位置坐标为:
\begin{align}\label{19.3}
  \begin{cases}
x_{FYj,t} = x_{FYj,0} + v_{FYj} t \cos\alpha \\
y_{FYj,t} = y_{FYj,0} + v_{FYj} t \sin\alpha \\
z_{FYi,t} = z_{FYj,0}
\end{cases}
\end{align}
从而可得\( t_{FYj,11} \) 时刻无人机 \( FYj \) 的位置坐标,即烟幕干扰弹投放点的位置坐标
\begin{align}
  \begin{cases}
x_{FYj,t_{FYj,11}} = x_{FYj,0} + v_{FYj} t_{FYj,11} \cos\alpha \\
y_{FYj,t_{FYj,11}} = y_{FYj,0} + v_{FYj} t_{FYj,11} \sin\alpha \\
z_{FYi,t_{FYj,11}} = z_{FYj,0}
\end{cases}
\end{align}
\item \textbf{烟幕干扰弹起爆点}:设无人机 $FYj $投放一枚烟幕干扰弹在无人机受领任务 \( t_{FYj,12} \) s 后起爆,由问题一求得的\eqref{1.9}可得3架无人机投放的烟雾干扰弹在$t_{FYj,12}$时刻即其起爆时的位置坐标:
\begin{align}\label{19.4}
\left\{ \begin{array}{l}
	x_{FYj,t_{FYj,12}}=x_{FYj,t_{FYj,11}}+v_{FYj}\left( t_{FYj,12}-t_{FYj,11} \right) \cos \alpha\\
	y_{FYj,t_{FYj,12}}=y_{FYj,t_{FYj,11}}+v_{FYj}\left( t_{FYj,12}-t_{FYj,11} \right) \sin \alpha\\
	z_{FYj,t_{FYj,12}}=z_{FYj,t_{FYj,11}}-\frac{g\left( t_{FYj,12}-t_{FYj,11} \right) ^2}{2}\\
\end{array} \right. 
\end{align}
\par 通过问题1中的\eqref{1.12},得到 $t$ 时刻无人机$FYj$投放的烟幕干扰弹形成的烟幕云团是否对目标进行遮挡的判断条件$\Delta _{FYj1}\left( x_l,y_l,z_s \right) $。代入\eqref{1.16}\eqref{1.17},得到:
\begin{align}\label{1.1713}
	\begin{aligned}
d_{1j} &= \sqrt{k_{1j}^2(x_{M1(t)} - x_l)^2 + k_{1j}^2(y_{M1(t)} - y_l)^2 + k_{1j}^2(z_{M1(t)} - z_s)^2}, \\
d_{2j} &= \sqrt{k_{2j}^2(x_{M1(t)} - x_1)^2 + k_{2j}^2(y_{M1(t)} - y_l)^2 + k_{2j}^2(z_{M1(t)} - z_s)^2},
\end{aligned}
\end{align}




并将其代入\eqref{1.19}中即
\begin{align}\label{1.98}
	\left\{ \begin{array}{l}
	\varDelta <0\ \ \ \text{未形成有效遮挡}\\
	\varDelta \ge 0\left\{ \begin{array}{l}
	\min \left\{ d_{1j},d_{2j} \right\} >\left| \overrightarrow{N1M1} \right|\ \ \ \ \ \ \ \ \text{未形成有效遮挡}\\
	\min \left\{ d_{1j},d_{2j} \right\} \le \left| \overrightarrow{N1M1} \right|\ \ \ \ \ \ \ \ \text{有效遮挡}\\
\end{array} \right.\\
\end{array} \right. 
\end{align}
\par 判断$t_w$时刻下无人机$FYj$释放的烟幕干扰弹形成的烟幕云团是否对真目标进行遮挡。又因为每个烟幕弹形成的云团都将在20s后消散,所以规定: 
\begin{align}\label{14.9}
	\Delta _{FYj1}\left( x_l,y_l,z_s \right) =\left\{ \begin{array}{l}
	-1,\,\, t\ge t_{FY1,12}+\Delta t_0\\
	-1,\,\, t\ge t_{FY2,12}+\Delta t_0\\
\end{array} \right.
\end{align}
\end{itemize}
最后,同理遍历真目标圆柱上的所有离散点$(x_l,y_l,z_s)$和离散时刻$t_w$,将所有对真目标构成有效遮挡的时刻累加起来就得到该条件下的$\Delta t_{31}$.

\noindent \textbf{Step 3 约束条件}
\begin{itemize}
\item \textbf{无人机的飞行速度}:由于无人机受领任务后,可根据需要瞬时调整飞行方向,然后以70-140m/s的速度等高度匀速直线飞行。因此:
\begin{align}\label{11.8}
  70 \leq v_{FY1} \leq 140
\end{align}
\end{itemize}
\textbf{Step 4 优化模型}
\par 综上所述,3架无人机$FYj$释放的烟幕遮挡弹的有效遮蔽总时间单目标优化模型为:
\begin{align}
  \begin{array}{c}
  \underset{\alpha_j ,v_{FYj},t_{FYj,j1},t_{FYj,j2}}{\max}\Delta t_{31}
  \\
\left\{ \begin{array}{l}
	t\text{时刻无人机的位置坐标:}\\
	\left\{ \begin{array}{l}
x_{FYj,t} = x_{FYj,0} + v_{FYj} t \cos\alpha \\
y_{FYj,t} = y_{FYj,0} + v_{FYj} t \sin\alpha \\
z_{FYi,t} = z_{FYj,0}\\
0\leq \alpha < 2\pi \\
\end{array} \right.\\
  70 \leq v_{FY1} \leq 140\\
	\text{无人机}FYj\text{投放的烟雾干扰弹起爆时的位置坐标:}\\
	\left\{ \begin{array}{l}
	x_{FYj,t_{FYj,12}}=x_{FYj,t_{FYj,11}}+v_{FYj}\left( t_{FYj,12}-t_{FYj,11} \right) \cos \alpha\\
	y_{FYj,t_{FYj,12}}=y_{FYj,t_{FYj,11}}+v_{FYj}\left( t_{FYj,12}-t_{FYj,11} \right) \sin \alpha\\
	z_{FYj,t_{FYj,12}}=z_{FYj,t_{FYj,11}}-\frac{g\left( t_{FYj,12}-t_{FYj,11} \right) ^2}{2}\\
\end{array} \right. 
\end{array} \right.  
\end{array}
\end{align}


  \end{document}