\documentclass[../main.tex]{subfiles}
\graphicspath{{figures/}{../figures/}}

\begin{document}
% \todo[color=green!40]{完成问题三模型的建立(sections/q5\_build)}
\noindent \textbf{Step 1 目标函数}
\par 本问中,我们的目标是通过设计5架无人机$FYj$与其分别投放的烟幕干扰弹相关参数,使得烟幕干扰弹对导弹$M1$,$M2$,$M3$的有效遮蔽总时间尽可能长,因此目标函数为:
\begin{align}\label{11.1}
  \underset{\alpha _j,v_{FYj},t_{FYj,i1},t_{FYj,i2}}{\max}\Delta t
\end{align}
\noindent \textbf{Step 2 决策变量}

\begin{itemize}
\item \textbf{无人机$FYj$的方向}:设$\alpha_j$为无人机$FYj$与$x$轴正方向的夹角,逆时针为正,范围为$\left[ 0,2\pi \right] $,从而确定无人机$FYi$的飞行方向$(i=1,2,3,4,5.j=1,2,3)$。

\item \textbf{来袭导弹}:由问题一同理可知导弹$Mk(k=1,2,3)$在$t$时刻的位置坐标$Mk\left( t \right)$为:
\begin{align}
	\left\{ \begin{array}{l}
	x_{Mk,t}=x_{Mk,0}-\frac{x_{Mk,0}v_0t}{\sqrt{x_{Mk,0}^{2}+y_{Mk,0}^{2}+z_{Mk,0}^{2}}}\\
	y_{Mk,t}=y_{Mk,0}-\frac{y_{Mk,0}v_0t}{\sqrt{x_{Mk,0}^{2}+y_{Mk,0}^{2}+z_{Mk,0}^{2}}}\\
	z_{Mk,t}=z_{Mk,0}-\frac{z_{Mk,0}v_0t}{\sqrt{x_{Mk,0}^{2}+y_{Mk,0}^{2}+z_{Mk,0}^{2}}}\\
\end{array} \right. 
\end{align}
\item  \textbf{无人机$FYj$的飞行速度} :设无人机的飞行速度为 \( v_{FYj} \),无人机受领任务后,保持等高度匀速直线运动。由第一问可知\( t \) 时刻无人机 \( FYj \) 的位置坐标为:
\begin{align}\label{11.3}
\begin{cases}
x_{FYj,t} = x_{FYj,0} + v_{FYj} t \cos\alpha \\
y_{FYj,t} = y_{FYj,0} + v_{FYj} t \sin\alpha \\
z_{FYj,t} = z_{FYj,0}
\end{cases}
\end{align}
\item \textbf{烟幕干扰弹投放点}:设无人机$FYj$在受领任务 \( t_{FYj,i1} \) s 后投放第$i$枚烟幕干扰弹。 由\eqref{11.3}式可知,无人机$FYj$投放的第$i$枚烟幕干扰弹的投放点位置坐标为
\begin{align}
\begin{cases}
x_{FYj,t_{FYj,i1}} = x_{FYj,0} + v_{FYj} t_{FYj,i1} \cos\alpha \\
y_{FYj,t_{FYj,i1}} = y_{FYj,0} + v_{FYj} t_{FYj,i1} \sin\alpha \\
z_{FYj,t_{FYj,i1}} = z_{FYj,0}
\end{cases}
\end{align}

\item \textbf{烟幕干扰弹起爆点}:设无人机 $FYj$ 投放的第$i$枚烟幕干扰弹在无人机受领任务 \( t_{FYj,i2} \) 秒后起爆,因此由问题一可得投放的第$i$枚烟雾干扰弹在$t_{FYj,i2}$时刻即其起爆时的位置坐标:
\begin{align}\label{19.4}
\left\{ \begin{array}{l}
	x_{FYji,t_{FYj,i2}}=x_{FYj,t_{FYj,i1}}+v_{FYj}\left( t_{FYj,i2}-t_{FYj,i1} \right) \cos \alpha\\
	y_{FYji,t_{FYj,i2}}=y_{FYj,t_{FYj,i1}}+v_{FYj}\left( t_{FYj,i2}-t_{FYj,i1} \right) \sin \alpha\\
	z_{FYji,t_{FYj,i2}}=z_{FYj,t_{FYj,i1}}-\frac{g\left( t_{FYj,i2}-t_{FYj,i1} \right) ^2}{2}\\
\end{array} \right. 
\end{align}
\par 代入问题1中,得到 $t$ 时刻$(t\in \left[ \underset{i=1,2,3,4,5}{\min}\left\{ t_{FYj,12} \right\} ,\underset{i=1,2,3,4,5}{\max}\left\{ t_{FYj,32} \right\} +\Delta t_0 \right]$)第$i$枚烟幕干扰弹形成的烟幕云团是否对目标进行遮挡,避免导弹$Mk$发现真目标的判别式$\Delta _{FYjik}\left( x_l,y_l,z_s \right) $。代入\eqref{1.16}\eqref{1.17},得到:
\begin{align}\label{1.1714}
	\begin{aligned}
d_{1ji} &= \sqrt{k_{1ji}^2(x_{Mk(t)} - x_l)^2 + k_{1ji}^2(y_{Mk(t)} - y_l)^2 + k_{1ji}^2(z_{Mk(t)} - z_s)^2}, \\
d_{2ji} &= \sqrt{k_{2ji}^2(x_{Mk(t)} - x_l)^2 + k_{2ji}^2(y_{Mk(t)} - y_1)^2 + k_{2ji}^2(z_{Mk(t)} - z_s)^2},
\end{aligned}
\end{align}
\par 其中导弹 \( M1 \) 到圆柱侧面上的点的的距离:
\begin{align}\label{1.182}
	\left| \overrightarrow{N1Mk} \right|=\sqrt{\left( x_{Mk(t)}-x_{n1} \right) ^2+\left( y_{Mk(t)}-y_{n1} \right) ^2+\left( y_{Mk(t)}-y_{n2} \right) ^2}
\end{align}












只有在$t$时刻,对于3枚导弹同时遮挡,才有可能看作有效遮挡,结合问题一中的\eqref{1.19},则此问中的有效遮挡判断为:


\begin{align}\label{1.88}
	\left\{ \begin{array}{l}
	\min \left\{ \Delta _{FYji1},\Delta _{FYji2},\Delta _{FYji3} \right\} <0\,\,\,\,\,\,\text{未形成有效遮挡}\\
	\min \left\{ \Delta _{FYji1},\Delta _{FYji2},\Delta _{FYji3} \right\} \ge 0\left\{ \begin{array}{l}
	\min \left\{ d_{1ji},d_{2ji} \right\} >\left| \overrightarrow{N1Mk} \right|\,\,\,\,\,\,\,\,\,\,\,\,\,\,\,\,\text{未形成有效遮挡}\\
	\min \left\{ d_{1ji},d_{2ji} \right\} \le \left| \overrightarrow{N1Mk} \right|\,\,\,\,\,\,\,\,\,\,\,\,\,\,\,\,\text{有效遮挡}\\
\end{array} \right.\\
\end{array} \right. 
\end{align}
判断$t$时刻下无人机$FYj$释放的第$i$枚烟幕干扰弹形成的烟幕云团是否对真目标进行遮挡。又因为每个烟幕干扰弹形成的云团都将在$\Delta t_0 = 20$秒后消散,所以规定: 
\begin{align}\label{14.907}
	\Delta _{FYjik}\left( x_l,y_l,z_s \right) =
	=-1\ \ \ \ \ \ \ \ \ \ t\ge t_{FYj,i2}+\Delta t_0\\
\end{align}
\end{itemize}




\textbf{Step 3 约束条件}
\begin{itemize}
\item \textbf{无人机的飞行速度}:由于无人机受领任务后,可根据需要瞬时调整飞行方向,然后以70-140m/s的速度等高度匀速直线飞行。因此:
\begin{align}\label{11.8}
  70 \leq v_{\text{FY1}} \leq 140
\end{align}
\item \textbf{无人机投放的烟幕干扰弹的时间}:
\par 由于题目要求每架无人机投放两枚烟幕干扰弹至少间隔1s,且据试验数据知,云团中心10m范围内的烟幕浓度在起爆20s内可为目标提供有效遮蔽。因此:
\begin{align}\label{12345}
\left\{ \begin{array}{l}
	t_{FYj,11}\in \left[ 0, \frac{x_{mk,0}}{v_0} \right]
	\\
	t_{FYj,12}\in \left[ t_{FYj,11}, \frac{d_{mk,0}}{v_0} \right]
	\\
	t_{FYj,21}\in \left[ t_{FYj,11}+1, \frac{dx_{mk,0}}{v_0} \right]
	\\
	t_{FYj,22}\in \left[ t_{FYj,21}, \frac{d_{mk,0}}{v_0} \right]
	\\
	t_{FYj,31}\in \left[ t_{FYj,21}+1, \frac{d_{mk,0}}{v_0} \right]
	\\
	t_{FYj,32}\in \left[ t_{FYj,21}, \frac{d_{mk,0}}{v_0} \right]
\end{array} \right. 
\end{align}
\end{itemize}
\textbf{Step 4 优化模型}
\par 综上所述,5架无人机释放的
相应的烟幕干扰弹对真目标的有效遮蔽总时间单目标优化模型为:
\begin{align}
  \begin{array}{c}
	   \underset{\alpha _j,v_{FYj},t_{FYj,i1},t_{FYj,i2}}{\max}\Delta t\\
\left\{ \begin{array}{l}
\text{导弹$Mk$在$t$时刻的位置坐标:}\\
	\left\{ \begin{array}{l}
	x_{Mk,t}=x_{Mk,0}-\frac{x_{Mk,0}v_0t}{\sqrt{x_{Mk,0}^{2}+y_{Mk,0}^{2}+z_{Mk,0}^{2}}}\\
	y_{Mk,t}=y_{Mk,0}-\frac{y_{Mk,0}v_0t}{\sqrt{x_{Mk,0}^{2}+y_{Mk,0}^{2}+z_{Mk,0}^{2}}}\\
	z_{Mk,t}=z_{Mk,0}-\frac{z_{Mk,0}v_0t}{\sqrt{x_{Mk,0}^{2}+y_{Mk,0}^{2}+z_{Mk,0}^{2}}}\\
\end{array} \right. \\
	t\text{时刻无人机的位置坐标:}\\
	\begin{cases}
x_{FYj,t} = x_{FYj,0} + v_{FYj} t \cos\alpha \\
y_{FYj,t} = y_{FYj,0} + v_{FYj} t \sin\alpha \\
z_{FYj,t} = z_{FYj,0}
	0\leq \alpha \leq 2\pi\\
\end{cases}\\
  70 \leq v_{\text{FY1}} \leq 140\\
	\text{第$j$架无人机的第$i$枚烟雾干扰弹起爆时的位置坐标:}\\
	\left\{ \begin{array}{l}
	x_{FYji,t_{FYj,i2}}=x_{FYj,t_{FYj,i1}}+v_{FYj}\left( t_{FYj,i2}-t_{FYj,i1} \right) \cos \alpha\\
	y_{FYji,t_{FYj,i2}}=y_{FYj,t_{FYj,i1}}+v_{FYj}\left( t_{FYj,i2}-t_{FYj,i1} \right) \sin \alpha\\
	z_{FYji,t_{FYj,i2}}=z_{FYj,t_{FYj,i1}}-\frac{g\left( t_{FYj,i2}-t_{FYj,i1} \right) ^2}{2}\\
\end{array} \right.  \\
	\left\{ \begin{array}{l}
	t_{FYj,11}\in \left[ 0, \frac{x_{mk,0}}{v_0} \right]
	\\
	t_{FYj,12}\in \left[ t_{FYj,11}, \frac{d_{mk,0}}{v_0} \right]
	\\
	t_{FYj,21}\in \left[ t_{FYj,11}+1, \frac{dx_{mk,0}}{v_0} \right]
	\\
	t_{FYj,22}\in \left[ t_{FYj,21}, \frac{d_{mk,0}}{v_0} \right]
	\\
	t_{FYj,31}\in \left[ t_{FYj,21}+1, \frac{d_{mk,0}}{v_0} \right]
	\\
	t_{FYj,32}\in \left[ t_{FYj,21}, \frac{d_{mk,0}}{v_0} \right]
\end{array} \right. 
\end{array} \right.  
\end{array}
\end{align}

\end{document}