\documentclass[../main.tex]{subfiles}
\graphicspath{{figures/}{../figures/}}

\begin{document}
% \todo[color=green!40]{完成问题四模型的求解(sections/q4\_solution)}

通过模拟物理退火过程的随机搜索与概率接受机制,在决策变量的可行域内寻找使有效遮蔽时间$\Delta t$最大化的最优解,具体步骤如下:

\noindent\textbf{步骤1 初始化参数}
\begin{itemize}
    \item \textbf{初始解生成}:在决策变量可行域内随机生成初始解$S_0=(\alpha_0, v_{\text{FY1},0})$,其中$\alpha_0 \in [0,2\pi]$,$v_{\text{FY1},0} \in [70,140]$;
    \item \textbf{初始温度$T_0$}:设定较高的初始温度(如$T_0=100$),确保算法初期能接受较差解,扩大搜索范围;
    \item \textbf{降温系数$k$}:设定降温速率(如$k=0.95$),控制温度随迭代逐步降低;
    \item \textbf{终止温度$T_{\text{end}}$}:设定停止阈值(如$T_{\text{end}}=10^{-5}$),当温度低于此值时终止迭代;
    \item \textbf{迭代次数$L$}:每轮温度下的迭代步数(如$L=50$),确保在当前温度下充分搜索邻域。
\end{itemize}

\noindent\textbf{步骤2 目标函数计算(核心步骤)}
对任意解$S=(\alpha, v_{\text{FY1}})$,计算其对应的有效遮蔽时间$\Delta t$,步骤如下:
\begin{itemize}
    \item \textbf{无人机运动模拟}:根据$\alpha$和$v_{\text{FY1}}$,计算无人机在投放时刻$t_1$的位置$(x_{\text{FY1},t_1}, y_{\text{FY1},t_1}, z_{\text{FY1},t_1})$;
    \item \textbf{烟幕弹起爆位置计算}:基于$t_2 = t_1 + \Delta t_{\text{delay}}$($\Delta t_{\text{delay}}$为烟幕弹飞行时间,固定参数),计算起爆位置$(x_{\text{FY11},t_2}, y_{\text{FY11},t_2}, z_{\text{FY11},t_2})$,其中$x$、$y$方向按无人机速度惯性运动($\lambda=\alpha$,与无人机同方向),$z$方向受重力下落;
    \item \textbf{烟幕云团位置随时间变化}:对$t \in [t_2, t_2+\Delta t_0]$,计算云团中心坐标$(x_{\text{FY11},t}, y_{\text{FY11},t}, z_{\text{FY11},t})$,其中$z$方向以$v_1$下沉;
    \item \textbf{真目标采样}:在圆柱面(真目标)上均匀采样若干点(如不同角度和高度),覆盖目标关键区域;
    \item \textbf{遮挡时间判定}:对每个采样点,结合导弹飞行轨迹(预设参数),通过判别式$\Delta \geq 0$判断$t$时刻是否遮挡,记录所有有效遮挡的时间区间,总时长即为$\Delta t$。
\end{itemize}

\noindent\textbf{步骤3 邻域解生成}

为当前解$S=(\alpha, v_{\text{FY1}})$生成邻域解$S'=(\alpha', v_{\text{FY1}}')$,确保新解在可行域内:
\begin{itemize}
    \item $\alpha' = \alpha + \Delta\alpha$,其中$\Delta\alpha$为随机扰动(如$\pm0.1$弧度),若$\alpha'$超出$[0,2\pi]$则取模调整;
    \item $v_{\text{FY1}}' = v_{\text{FY1}} + \Delta v$,其中$\Delta v$为随机扰动(如$\pm5$ m/s),若$v_{\text{FY1}}'$超出$[70,140]$则截断至边界。
\end{itemize}

\noindent\textbf{步骤4 判断准则(接受/拒绝新解)}
\begin{itemize}
    \item 计算新解与当前解的目标函数差值:$\Delta E = \Delta t(S') - \Delta t(S)$;
    \item 若$\Delta E > 0$(新解更优):直接接受$S'$作为当前解;
    \item 若$\Delta E \leq 0$(新解较差):以概率$P = \exp\left(\frac{\Delta E}{T}\right)$接受$S'$,其中$T$为当前温度。温度越高,接受较差解的概率越大,利于跳出局部最优。
\end{itemize}

\noindent\textbf{步骤5 降温与迭代}
\begin{itemize}
    \item 每完成$L$次迭代后,按$T = k \cdot T$降低温度;
    \item 重复“邻域搜索→接受准则→降温”过程,直至温度$T \leq T_{\text{end}}$。
\end{itemize}

\noindent\textbf{步骤6 终止与最优解输出}

迭代终止后,输出历史最优解$S^*=(\alpha^*, v_{\text{FY1}}^*)$及其对应的最大有效遮蔽时间$\Delta t^*$。

按照上述算法思路,利用Python求解得

\begin{table}[H]
\caption{标准三线表格}
\label{tab:001} 
\centering
\begin{small}
\begin{tabular}{ccccc}
\toprule[1.5pt]
无人机编号 &无人机运动方向 & 无人机运动速度  & 烟幕干扰弹投放点的x坐标& 烟幕干扰弹投放点的y坐标 \\
\midrule[1pt]
 &             &                      & 1.79                    & 0.04089     \\            
 &             &                       & 1.79                    & 0.04089      \\           
 &             &                     & 1.79                    & 0.04089      \\           
 &             &                     & 1.79                    & 0.04089       \\          
 &             &                    & 1.79                    & 0.04089      \\            
\bottomrule[1.5pt]
\end{tabular}
\end{small}
\end{table}


\begin{table}[H]
\caption{标准三线表格}
\label{tab:031} 
\centering
\begin{small}
\begin{tabular}{ccccc}
\toprule[1.5pt]
    干扰弹投放点的z坐标 &干扰弹起爆点的x坐标&干扰弹起爆点的y坐标&干扰弹起爆点的z坐标&有效干扰时长\\
\midrule[1pt]
             &                   & 0.000674     & 1.79                    & 0.04089  \\               
             &                   & 0.000674     & 1.79                    & 0.04089  \\               
             &                   & 0.000674     & 1.79                    & 0.04089  \\                
             &                   & 0.000674     & 1.79                    & 0.04089  \\                
             &                   & 0.000674     & 1.79                    & 0.04089  \\               
\bottomrule[1.5pt]
\end{tabular}
\end{small}
\end{table}






















\end{document}