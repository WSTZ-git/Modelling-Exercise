\documentclass[../main.tex]{subfiles}
\graphicspath{{figures/}{../figures/}}

\begin{document}
% \todo[color=green!30]{完成问题分析(sections/analyse)}
\subsection{问题一的分析}
\par 问题一要我们求解在给定条件下,烟雾干扰弹对导弹 \( M1 \) 的有效遮挡时长。因此我们需要知道烟雾干扰弹形成的云团球体在某时刻下是否对导弹进行遮挡,即导弹与假目标上任一点的连线是否与云团球体相交。首先,对于某时刻下导弹与真目标上任一点的连线方程,由导弹在某时刻下直角坐标系下的位置坐标与真目标所在圆柱侧面上任一点确定。我们已知导弹 \( M1 \) 和假目标在题目给定坐标系下的初始位置,以及导弹直指假目标的飞行方向和飞行速度,因此得到导弹 \( M1 \) 在 某时刻下的坐标,通过题目给定条件,可以得到真目标所在的圆柱侧面各点的位置,由此可知某时刻下导弹与真目标上任一点的连线方程;
其次,对于云团球体所在的球面方程,由云团中心和半径确定,因此需要知晓在某时刻下云团中心的位置坐标,我们已知无人机 \( FY1 \) 的初始位置,无人机匀速直线运动至假目标的速度,进而得到烟幕干扰弹投放时的位置,已知烟幕干扰弹在重力作用下运动,从而得到烟幕干扰弹爆炸时的位置,也就是云团中心的初始位置,已知云团匀速下沉的速度,可以得到在 某时刻云团球心的位置,进而得到云团的球面方程;
最后,将某时刻导弹\( M1 \)与真目标上任一点的连线方程与云团球体在某时刻下的球面方程进行联立,可以判断在某时刻下云团是否对导弹进行遮挡,从而得到烟幕干扰弹对导弹的有效遮挡时长。
\subsection{问题二的分析}




\subsection{问题三的分析}




\subsection{问题四的分析}




\subsection{问题五的分
析}  










\end{document}