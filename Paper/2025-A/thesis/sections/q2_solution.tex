\documentclass[../main.tex]{subfiles}
\graphicspath{{figures/}{../figures/}}

\begin{document}
<<<<<<< HEAD
  % % \todo[color=green!30]{完成问题二模型的求解(sections/q2\_solution)}
=======
% \todo[color=green!30]{完成问题二模型的求解(sections/q2\_solution)}

遍历算法通过对决策变量可行域网格化离散,枚举所有离散解并计算有效遮蔽时间$\Delta t$,最终筛选最优解,核心步骤如下:

\noindent\textbf{步骤1 明确优化模型核心要素}
\begin{itemize}
    \item \textbf{目标函数}:$\max \Delta t$(导弹路径与烟幕有效球形区域($r=2.5$m)的交集时长);
    \item \textbf{决策变量}:
          \begin{itemize}
              \item 飞行方向角$\alpha \in [0,2\pi]$;
              \item 飞行速度$v_{\text{FY1}} \in [70,140]$ m/s;
              \item 投放时刻$t_1 \in [0,10]$ s;
              \item 弹飞行时间$\Delta t_{\text{delay}}=t_2-t_1 \in [1,5]$ s。
          \end{itemize}
    \item \textbf{关键约束}:
          \begin{itemize}
              \item 无人机位置:$x_{\text{FY1},t}=x_{\text{FY1},0}+v_{\text{FY1}}t\cos\alpha$,$y_{\text{FY1},t}=y_{\text{FY1},0}+v_{\text{FY1}}t\sin\alpha$,$z_{\text{FY1},t}=z_{\text{FY1},0}$;
              \item 烟幕起爆位置:$x_{\text{FY11},t_2}=x_{\text{FY1},t_1}-v_{\text{FY1}}\Delta t_{\text{delay}}\cos\lambda$,$y_{\text{FY11},t_2}=y_{\text{FY1},t_1}-v_{\text{FY1}}\Delta t_{\text{delay}}\sin\lambda$,$z_{\text{FY11},t_2}=z_{\text{FY1},t_1}-\frac{1}{2}g\Delta t_{\text{delay}}^2$;
              \item 烟幕云团位置:$x_{\text{FY11},t}=x_{\text{FY11},t_2}$,$y_{\text{FY11},t}=y_{\text{FY11},t_2}$,$z_{\text{FY11},t}=z_{\text{FY11},t_2}-v_1(t-t_2)$($t \in [t_2,t_2+20]$s);
              \item 真目标圆柱面:$x_1^2+(y_1-y_0)^2=r_0^2$,$z_1 \in [0,h_0]$($y_0=200$,$r_0=7$,$h_0=10$);
              \item 遮挡判断:$\Delta \geq 0$为遮挡,$\Delta <0$为未遮挡。
          \end{itemize}
\end{itemize}

\noindent\textbf{步骤2 决策变量网格化离散}

对连续变量按固定步长离散,生成网格点:
\begin{itemize}
    \item $\alpha$:步长$\Delta\alpha=\pi/20$,离散为$\alpha_k=k\cdot\Delta\alpha$($k=0,1,\dots,39$);
    \item $v_{\text{FY1}}$:步长$\Delta v=5$m/s,离散为$v_m=70+m\cdot\Delta v$($m=0,1,\dots,14$);
    \item $t_1$:步长$\Delta t_1=0.5$s,离散为$t_{1,n}=0+n\cdot\Delta t_1$($n=0,1,\dots,20$);
    \item $\Delta t_{\text{delay}}$:步长$\Delta t_d=0.5$s,离散为$\Delta t_{d,p}=1+p\cdot\Delta t_d$($p=0,1,\dots,8$)。
\end{itemize}

\noindent\textbf{步骤3 枚举所有离散解}

构建所有决策变量网格点的组合,生成离散解集合:
$$S = \{ S_{k,m,n,p}=(\alpha_k, v_m, t_{1,n}, \Delta t_{d,p}) \mid k=0,\dots,39;\ m=0,\dots,14;\ n=0,\dots,20;\ p=0,\dots,8 \}$$
总组合数为$40 \times 15 \times 21 \times 9 = 113400$组。\\[8pt]

\noindent\textbf{步骤4 计算每个解的$\Delta t$}

对任意$S_{k,m,n,p}$,按以下步骤计算$\Delta t$:
\begin{itemize}
    \item 算无人机投放位置:代入$t=t_{1,n}$、$\alpha=\alpha_k$、$v=v_m$,得$(x_{\text{FY1},t_1}, y_{\text{FY1},t_1}, z_{\text{FY1},t_1})$;
    \item 算烟幕起爆位置:$t_2=t_{1,n}+\Delta t_{d,p}$,代入起爆位置公式,得$(x_{\text{FY11},t_2}, y_{\text{FY11},t_2}, z_{\text{FY11},t_2})$;
    \item 算烟幕实时位置:对$t \in [t_2, t_2+20]$s(步长$0.1$s),得$(x_{\text{FY11},t}, y_{\text{FY11},t}, z_{\text{FY11},t})$;
    \item 真目标采样与遮挡判断:在圆柱面取50个采样点,对每个$t$和采样点,验证$\Delta \geq 0$是否成立;
    \item 统计$\Delta t$:合并遮挡时间区间,总时长即为该解的$\Delta t$。
\end{itemize}

\noindent\textbf{步骤5 筛选最优解}

比较所有离散解的$\Delta t$,输出最大$\Delta t^*$及对应最优决策变量:
$$(\alpha^*, v_{\text{FY1}}^*, t_1^*, t_2^*) = \mathrm{arg}\max _{S_{k,m,n,p}} \Delta t(S_{k,m,n,p}).$$

\noindent\textbf{步骤6 局部精度优化}

对$\alpha^*$、$v_{\text{FY1}}^*$附近区域缩小步长(如$\Delta\alpha=\pi/40$、$\Delta v=2$m/s),再次遍历,提升最优解精度。













>>>>>>> 75008762ebc888dad4df4a11cfa0cf2b63727c86



\end{document}