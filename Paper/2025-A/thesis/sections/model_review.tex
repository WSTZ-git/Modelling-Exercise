\documentclass[../main.tex]{subfiles}
\graphicspath{{figures/}{../figures/}}

\begin{document}
% % \todo[color=green!30]{完成模型评价(sections/model\_review)}
\subsection{模型的优点\cite{doubao}}
\noindent\textbf{1.}在模型求解过程中,针对不同问题采用了合适的优化算法。如问题二和问题三使用模拟退火算法,问题五采用先逐个优化单无人机投放策略再合并的方法。这些算法有助于在复杂的参数空间中寻找最优解,提高了模型的实用性和有效性。模拟退火算法通过模拟物理退火过程,能够在一定程度上避免陷入局部最优解,在问题三的求解中,通过设置初始温度、降温系数等参数,在决策变量的可行域内搜索使有效遮蔽时间最大化的最优解。
\\
\textbf{2.}在问题四的求解中,对导弹飞行速度进行了灵敏度分析。通过调整导弹飞行速度,重新优化无人机投弹策略并观察有效时间的变化,这有助于了解模型中关键参数的变化对结果的影响,为进一步优化模型和制定策略提供了参考依据。发现有效时间对导弹飞行速度敏感且随速度增加而减小,这对于根据导弹实际速度调整无人机干扰策略具有重要意义。


\subsection{模型的缺点\cite{doubao}}
\noindent\textbf{1.}计算复杂度较高:随着问题规模的增加,模型的复杂度迅速上升。在问题五中,需要考虑5架无人机、每架最多3枚烟幕弹以及3枚导弹的情况,决策变量众多,计算量巨大。这不仅增加了计算时间和资源消耗,还可能导致在实际应用中难以快速得到结果,限制了模型的实时性和应用范围。
\\
\textbf{2.}未充分考虑动态因素:模型主要基于固定的参数和静态的场景进行构建,对于战场环境中的动态因素考虑不足。战场上,导弹和无人机的运动方向、速度可能随时发生变化,烟幕云团的扩散和消散过程也可能受到外界因素干扰而动态变化。模型未能有效处理这些动态变化,使其在面对复杂多变的实际战场情况时适应性较差 。 



\end{document}