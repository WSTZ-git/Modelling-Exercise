\documentclass[../main.tex]{subfiles}
\graphicspath{{figures/}{../figures/}}

\begin{document}
% \todo[color=green!40]{完成问题三模型的建立(sections/q3\_build)}
\noindent \textbf{Step 1 目标函数}
\par 在本问中,我们的目标是通过设计无人机$FY1$与其投放的3枚烟幕干扰弹相关参数,使得3枚烟幕干扰弹对导弹$M1$的有效遮蔽总时间尽可能长,因此目标函数为:
\begin{align}\label{12.1}
  \underset{\alpha ,t_1,v_{FY1},t_2,i}{\max}\Delta t_{13}
\end{align}
\noindent \textbf{Step 2 决策变量}

\begin{itemize}
\item \textbf{无人机$FY1$的方向}:设$\alpha $为无人机$FY1$与$x$轴正方向的夹角,逆时针为正,范围为$\left[ 0,2\pi \right] $,从而确定了无人机$FY1$的飞行方向。
\item \textbf{烟幕干扰弹投放点}:设无人机$FY1$在受领任务 \( t_{FY1,i1} \) s 后投放第i枚烟幕干扰弹$(i=1,2,3)$。
\item  \textbf{无人机$FY1$的飞行速度} :设无人机的飞行速度为 \( v_{FY1} \),无人机受领任务后,保持等高度匀速直线运动。由问题一求得的\eqref{1.8}可知 \( t \) 时刻无人机 \( FY1 \) 的位置坐标为:
\begin{align}\label{12.3}
  \begin{cases}
x_{FY1,t} = x_{FY1,0} + v_{FY1} t \cos\alpha \\
y_{FY1,t} = y_{FY1,0} + v_{FY1} t \sin\alpha \\
z_{FY1,t} = z_{FY1,0}
\end{cases}
\end{align}
\item \textbf{烟幕干扰弹起爆点}:设无人机 FY1 投放的第$i$枚烟幕干扰弹在无人机受领任务 \( t_{FY1,i2} \) s 后起爆,根据问题1中的\eqref{1.9}得到投放的第$i$枚烟雾干扰弹在$t_{FY1,i2}$时刻即其起爆时的位置坐标:
\begin{align}\label{12.4}
\left\{ \begin{array}{l}
	x_{FY1i,t_{FY1,i2}}=x_{FY1,t_{FY1,i1}}+v_{FY1}\left( t_{FY1,i2}-t_{FY1,i1} \right) \cos \alpha\\
	y_{FY1i,t_{FY1,i2}}=y_{FY1,t_{FY1,i1}}+v_{FY1}\left( t_{FY1,i2}-t_{FY1,i1} \right) \sin \alpha\\
	z_{FY1i,t_{FY1,i2}}=z_{FY1,t_{FY1,i1}}-\frac{g\left( t_{FY1,i2}-t_{FY1,i1} \right) ^2}{2}\\
\end{array} \right. 
\end{align}

\par 通过问题1中的\eqref{1.12},得到 $t$ 时刻无人机$FY1$投放的第$i$枚烟幕干扰弹形成的烟幕云团是否对目标进行遮挡的判断条件$\Delta _{FY1i}\left( x_l,y_l,z_s \right) $,代入\eqref{1.16}\eqref{1.17},得到:
\begin{align}\label{1.1712}
	\begin{aligned}
d_{1i} &= \sqrt{k_{1i}^2(x_{M1(t)} - x_l)^2 + k_{1i}^2(y_{M1(t)} - y_l)^2 + k_{1i}^2(z_{M1(t)} - z_s)^2}, \\
d_{2i} &= \sqrt{k_{2i}^2(x_{M1(t)} - x_l)^2 + k_{2i}^2(y_{M1(t)} - y_1)^2 + k_{2i}^2(z_{M1(t)} - z_s)^2},
\end{aligned}
\end{align}



并将其代入\eqref{1.19}中即
\begin{align}\label{1.25657}
	\left\{ \begin{array}{l}
	\varDelta <0\ \ \ \text{未形成有效遮挡}\\
	\varDelta \ge 0\left\{ \begin{array}{l}
	\min \left\{ d_{1i},d_{2i} \right\} >\left| \overrightarrow{N1M1} \right|\ \ \ \ \ \ \ \ \text{未形成有效遮挡}\\
	\min \left\{ d_{1i},d_{2i} \right\} \le \left| \overrightarrow{N1M1} \right|\ \ \ \ \ \ \ \ \text{有效遮挡}\\
\end{array} \right.\\
\end{array} \right. 
\end{align}
\par 判断$t$时刻下无人机FY1释放的第$i$枚烟幕干扰弹形成的烟幕云团是否对真目标进行遮挡。又因为每个烟幕干扰弹形成的云团都将在$\Delta t_0 = 20$秒后消散,所以规定: 
\begin{align}\label{14.9}
\Delta _{FY1i}\left( x_l,y_l,z_s \right) =\left\{ \begin{array}{l}
-1,\,\,t\ge t_{FY1,i2}+\Delta t_0\\
-1,\,\,t\ge t_{FY1,i2}+\Delta t_0\\
\end{array} \right.
\end{align}
\end{itemize}

\textbf{Step 3 约束条件}
\begin{itemize}
\item \textbf{无人机的飞行速度}:由于无人机受领任务后,可根据需要瞬时调整飞行方向,然后以70-140m/s的速度等高度匀速直线飞行。因此:
\begin{align}\label{15.8}
  70 \leq v_{FY1} \leq 140
\end{align}
\item \textbf{无人机投放的烟幕干扰弹的时间}:由于题目要求每架无人机投放两枚烟幕干扰弹至少间隔1s,且据试验数据知,云团中心10m范围内的烟幕浓度在起爆20s内可为目标提供有效遮蔽。因此:
\begin{align}
\begin{cases}
t_{FY1,11}\in \left[ 0,\frac{x_{m1,0}}{v_0} \right]\\
t_{FY1,12}\in \left[ t_{FY1,11},\frac{d_{m1,0}}{v_0} \right]\\
t_{FY1,21}\in \left[ t_{FY1,11}+1,\frac{dx_{m1,0}}{v_0} \right]\\
t_{FY1,22}\in \left[ t_{FY1,21},\frac{d_{m1,0}}{v_0} \right]\\
t_{FY1,31}\in \left[ t_{FY1,21}+1,\frac{d_{m1,0}}{v_0} \right]\\
t_{FY1,32}\in \left[ t_{FY1,21},\frac{d_{m1,0}}{v_0} \right]\\
\end{cases}
\end{align}
\end{itemize}
\textbf{Step 4 优化模型}
\par 综上所述,无人机$FY1$释放的3枚烟幕干扰弹有效遮蔽时间单目标优化模型为:
\begin{align}
  \begin{array}{c}
\underset{\alpha ,t_1,v_{FY1},t_2,i}{\max}\Delta t_{13}
  \\
\left\{ \begin{array}{l}
	t\text{时刻无人机的位置坐标:}\\
	\left\{ \begin{array}{l}
	x_{FY1,t}=x_{FY1,0}+v_{FY1}t\cos \alpha \\
	y_{FY1,t}=y_{FY1,0}+v_{FY1}t\sin \alpha \\
	z_{FY1,t}=z_{FY1,0}\\
	0\leq \alpha < 2\pi \\
\end{array} \right.\\
  70 \leq v_{FY1} \leq 140\\
	\text{第$i$枚烟雾干扰弹起爆时的位置坐标:}\\
	\left\{ \begin{array}{l}
	x_{FY1i,t_{FY1,i2}}=x_{FY1,t_{FY1,i1}}+v_{FY1}\left( t_{FY1,i2}-t_{FY1,i1} \right) \cos \alpha\\
	y_{FY1i,t_{FY1,i2}}=y_{FY1,t_{FY1,i1}}+v_{FY1}\left( t_{FY1,i2}-t_{FY1,i1} \right) \sin \alpha\\
	z_{FY1i,t_{FY1,i2}}=z_{FY1,t_{FY1,i1}}-\frac{g\left( t_{FY1,i2}-t_{FY1,i1} \right) ^2}{2}\\
\end{array} \right. \\
	\left\{ \begin{array}{l}
	t_{FY1,11}\in \left[ 0, \frac{x_{m1,0}}{v_0} \right]
	\\t_{FY1,12}\in \left[ t_{FY1,11}, \frac{d_{m1,0}}{v_0} \right]
	\\t_{FY1,21}\in \left[ t_{FY1,11}+1, \frac{dx_{m1,0}}{v_0} \right]
	\\t_{FY1,22}\in \left[ t_{FY1,21}, \frac{d_{m1,0}}{v_0} \right]
	\\t_{FY1,31}\in \left[ t_{FY1,21}+1, \frac{d_{m1,0}}{v_0} \right]
	\\t_{FY1,32}\in \left[ t_{FY1,21}, \frac{d_{m1,0}}{v_0} \right]
\end{array} \right. 
\end{array} \right.  
\end{array}
\end{align}

  \end{document}