\documentclass[../main.tex]{subfiles}
\graphicspath{{figures/}{../figures/}}

\begin{document}
% \todo[color=green!40]{完成问题三模型的建立(sections/q3\_build)}
\noindent \textbf{Step 1 }
\begin{itemize}
\item \textbf{目标函数}
\end{itemize}
根据题意,以无人机FY1的飞行方向与$x$轴正向的夹角$\alpha$,飞行速度$v_{FY1}$,三枚烟幕干扰弹的投放时间$t_{11}$、$t_{21}$、$t_{31}$,三枚烟幕干扰弹的起爆时间$t_{12}$、$t_{22}$、$t_{32}$为变量.建立单目标优化模型,目标函数为
\begin{align}
\underset{\alpha ,v_{FY1},t_{11},t_{12},t_{21},t_{22},t_{31},t_{32}}{\mathrm{Arg}\max}\Delta t=.
\end{align}
\\
\noindent \textbf{Step 2 }
\begin{itemize}
\item \textbf{计算有效遮挡时间}
\end{itemize}
\par \textbf{(1) }
由问题1同理可得,无人机FY1第$i(i=1,2,3)$个投放的烟雾干扰弹在投放时的在$t$时刻的位置坐标
\begin{align}
\begin{cases}
x_{FY1,t_{i1}}=x_{FY1,0}-v_{FY1}t_{i1}\cos \alpha\\
y_{FY1,t_{i1}}=y_{FY1,0}-v_{FY1}t_{i1}\sin \alpha\\
z_{FY1,t_{i1}}=z_{FY1,0}\\
\end{cases}
\end{align}
以及无人机FY1

\par \textbf{(2) }
\par \textbf{(3) }
\\
\noindent \textbf{Step 3 }
\begin{itemize}
\item \textbf{约束条件}
\end{itemize}
\par \textbf{(1) }
\par \textbf{(2) }
\par \textbf{(3) }
   

  \end{document}