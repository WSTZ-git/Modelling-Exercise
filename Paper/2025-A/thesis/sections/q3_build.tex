\documentclass[../main.tex]{subfiles}
\graphicspath{{figures/}{../figures/}}

\begin{document}
% \todo[color=green!40]{完成问题三模型的建立(sections/q3\_build)}
\noindent \textbf{Step 1 目标函数}
\par 在本问中,我们的目标是通过设计无人机$FY1$与其投放的3枚烟幕干扰弹相关参数,使得3枚烟幕干扰弹对导弹$M1$的有效遮蔽总时间尽可能长,因此目标函数为:
\begin{align}\label{12.1}
  \max \Delta t_{13}
\end{align}
\noindent \textbf{Step 2 决策变量}

\begin{itemize}
\item \textbf{无人机$FY1$的方向}:设$\alpha $为无人机$FY1$与$x$轴正方向的夹角,范围为$\left[ 0,2\pi \right] $,从而确定了无人机$FY1$的飞行方向。
\item \textbf{烟幕干扰弹投放点}:设无人机$FY1$在受领任务 \( t_{FY1,i1} \) s 后投放第i枚烟幕干扰弹$(i=1,2,3)$。
\item  \textbf{无人机$FY1$的飞行速度} :设无人机的飞行速度为 \( v_{FY1} \),无人机受领任务后,保持等高度匀速直线运动。由问题一求得的\eqref{1.8}可知 \( t \) 时刻无人机 \( FY1 \) 的位置坐标为:
\begin{align}\label{12.3}
  \begin{cases}
x_{FY1,t} = x_{FY1,0} + v_{FY1} t \cos\alpha \\
y_{FY1,t} = y_{FY1,0} + v_{FY1} t \sin\alpha \\
z_{FY1,t} = z_{FY1,0}
\end{cases}
\end{align}
\item \textbf{烟幕干扰弹起爆点}:设无人机 FY1 投放的第$i$枚烟幕干扰弹在无人机受领任务 \( t_{FY1,i2} \) s 后起爆,根据问题1中的\eqref{1.9}得到投放的第$i$枚烟雾干扰弹在$t_{FY1,i2}$时刻即其起爆时的位置坐标:
\begin{align}\label{12.4}
\left\{ \begin{array}{l}
	x_{FY1i,t_{FY1,i2}}=x_{FY1,t_{FY1,i1}}+v_{FY1}\left( t_{FY1,i2}-t_{FY1,i1} \right) \cos \alpha\\
	y_{FY1i,t_{FY1,i2}}=y_{FY1,t_{FY1,i1}}+v_{FY1}\left( t_{FY1,i2}-t_{FY1,i1} \right) \sin \alpha\\
	z_{FY1i,t_{FY1,i2}}=z_{FY1,t_{FY1,i1}}-\frac{g\left( t_{FY1,i2}-t_{FY1,i1} \right) ^2}{2}\\
\end{array} \right. 
\end{align}

\par 代入问题1中的\eqref{1.10}\eqref{1.11},得到 $t$ 时刻无人机$FY1$投放的第$i$枚烟幕干扰弹形成的烟幕云团是否对目标进行遮挡的判断条件$\Delta _{FY1i}\left( x_1,y_1,z_1 \right) $.将真目标所在圆柱侧面上任一点坐标$(x_1, y_1, z_1)$ 带入,当 $\Delta _{FY1i} \geq 0$ 时,第$i$枚烟幕干扰弹形成的烟幕云团对目标进行遮挡,当 $\Delta _{FY1i} < 0$ 时,第$i$枚烟幕干扰弹形成的烟幕云团未对目标形成遮挡。并规定: 
\begin{align}\label{14.9}
	\Delta _{FY1i}\left( x_1,y_1,z_1 \right) =\left\{ \begin{array}{l}
	=-1\ \ \ \ \ \ \ \ \ \ t\ge t_{12}+\Delta t_0\\
	=-1\ \ \ \ \ \ \ \ \ \ t\ge t_{22}+\Delta t_0\\
\end{array} \right.
\end{align}





我们记无人机$FY1$投放的第$i$枚烟幕干扰弹形成的烟幕云团对目标形成遮挡为$a_{i}^{1}$,其为0,1向量,则
\begin{align}
a_{i}^{1}=\begin{cases}
	0\ \ \ \Delta _{FY1i}\geq 0&		\text{未遮挡}\\
	1\ \ \ \Delta _{FY1i}<0&		\text{遮挡}\\
\end{cases}
\end{align}\label{11.7}
\par 对于无人机 FY1 投放的3枚烟幕干扰弹只要有一枚形成的烟幕云团对目标进行遮挡即可,因此
\begin{align}
\sum_{i=1}^{3} a_{i}^{1} = 
\begin{cases} 
0 & \text{未遮挡} \\
\text{else} & \text{遮挡}
\end{cases}
\end{align}




\end{itemize}




\textbf{Step 3 约束条件}
\begin{itemize}
\item \textbf{无人机的飞行速度}:由于无人机受领任务后,可根据需要瞬时调整飞行方向,然后以70-140m/s的速度等高度匀速直线飞行。因此:
\begin{align}\label{15.8}
  70 \leq v_{\text{FY1}} \leq 140
\end{align}
\item \textbf{无人机投放的烟幕干扰弹的时间}:我们假定当导弹 \( M_1 \) 匀速飞行至假目标时,发现该目标为假目标。已知该导弹飞行速度 \( v_0 \),以及初始坐标 \( M_{1,0}(x_{m1,0}, y_{m1,0}, z_{m1,0}) \) 和假目标位置坐标(即原点 \( O \))。因此警戒雷达发现导弹后 \( t \) 时刻满足范围为:
\begin{align}
\begin{cases}
d_{m1,0} = \sqrt{(x_{m1,0})^2 + (z_{m1,0})^2} \\
0 \leq t \leq \frac{d_{m1,0}}{v_0}
\end{cases}
\end{align}
\par 由于题目要求每架无人机投放两枚烟幕干扰弹至少间隔1s,且据试验数据知,云团中心10m范围内的烟幕浓度在起爆20s内可为目标提供有效遮蔽。因此:
\begin{align}
\left\{ \begin{array}{l}
	t_{FY1,11}\in \left[ 0, \frac{x_{m1,0}}{v_0} \right]
	\\\\
	t_{FY1,12}\in \left[ t_{FY1,11}, \frac{d_{m1,0}}{v_0} \right]
	\\\\
	t_{FY1,21}\in \left[ t_{FY1,11}+1, \frac{dx_{m1,0}}{v_0} \right]
	\\\\
	t_{FY1,22}\in \left[ t_{FY1,21}, \frac{d_{m1,0}}{v_0} \right]
	\\\\
	t_{FY1,31}\in \left[ t_{FY1,21}+1, \frac{d_{m1,0}}{v_0} \right]
	\\\\
	t_{FY1,32}\in \left[ t_{FY1,21}, \frac{d_{m1,0}}{v_0} \right]
\end{array} \right. 
\end{align}
\end{itemize}
\textbf{Step 4 优化模型}
\par 综上所述,有效遮蔽时间单一目标优化模型为:
\begin{align}
  \begin{array}{c}
	 \max \Delta t
  \\
\left\{ \begin{array}{l}
	t\text{时刻无人机的位置坐标:}\\
	\left\{ \begin{array}{l}
	x_{FY1,t}=x_{FY1,0}+v_{FY1}t\cos \alpha \\
	y_{FY1,t}=y_{FY1,0}+v_{FY1}t\sin \alpha \\
	z_{FY1,t}=z_{FY1,0}\\
\end{array} \right.\\
	0\leq \alpha \leq 2\pi \\
  70 \leq v_{\text{FY1}} \leq 140\\
	\text{第$i$枚烟雾干扰弹起爆时的位置坐标:}\\
	\left\{ \begin{array}{l}
	x_{FY1i,t_{FY1,i2}}=x_{FY1,t_{FY1,i1}}+v_{FY1}\left( t_{FY1,i2}-t_{FY1,i1} \right) \cos \alpha\\
	y_{FY1i,t_{FY1,i2}}=y_{FY1,t_{FY1,i1}}+v_{FY1}\left( t_{FY1,i2}-t_{FY1,i1} \right) \sin \alpha\\
	z_{FY1i,t_{FY1,i2}}=z_{FY1,t_{FY1,i1}}-\frac{g\left( t_{FY1,i2}-t_{FY1,i1} \right) ^2}{2}\\
\end{array} \right. \\
	t\text{时刻第$i$枚烟幕干扰弹形成的云团中心坐标:}\\
	\left\{ \begin{array}{l}
	x_{FY11,t}=x_{FY11,t_{FY1,i2}}\\
	y_{FY11,t}=y_{FY11,t_{FY1,i2}}\\
	z_{FY11,t}=z_{FY11,t_{FY1,i2}}-v_1\left( t-t_{FY1,i2} \right)\\
	t\in \left[ t_{FY1,i2},t_{FY1,i2}+\Delta t_0 \right]
\end{array} \right.\\
	\text{第$i$枚烟幕干扰弹形成的云团球体球面方程:}\\
	O_{FY1i,t}:\left( x-x_{FY1i,t} \right) ^2+\left( y-y_{FY1i,t} \right) ^2+\left( z-z_{FY1i,t} \right) ^2=r^2
	\\
	\text{圆柱面上点坐标:}\\
	\left\{ \begin{array}{c}
	\begin{array}{l}
	x_{1}^{2}+\left( y_{1}^{2}-y_0 \right) ^2=r_{0}^{2}\\
	z_1\in \left[ 0,h_0 \right]\\
\end{array}\\
\end{array} \right.\\
	\text{烟幕云团遮挡判断条件:}\\
	\sum_{i=1}^{3} a_{i}^{1} = 
\begin{cases} 
0 & \text{未遮挡} \\
\text{else} & \text{遮挡}
\end{cases}
\\
\left\{ \begin{array}{l}
	t_{FY1,11}\in \left[ 0, \frac{x_{m1,0}}{v_0} \right]
	\\t_{FY1,12}\in \left[ t_{FY1,11}, \frac{d_{m1,0}}{v_0} \right]
	\\t_{FY1,21}\in \left[ t_{FY1,11}+1, \frac{dx_{m1,0}}{v_0} \right]
	\\t_{FY1,22}\in \left[ t_{FY1,21}, \frac{d_{m1,0}}{v_0} \right]
	\\t_{FY1,31}\in \left[ t_{FY1,21}+1, \frac{d_{m1,0}}{v_0} \right]
	\\t_{FY1,32}\in \left[ t_{FY1,21}, \frac{d_{m1,0}}{v_0} \right]
\end{array} \right. 
\end{array} \right.  
\end{array}
\end{align}

  \end{document}