\documentclass[../main.tex]{subfiles}
\graphicspath{{figures/}{../figures/}}

\begin{document}
% \todo[color=green!40]{完成问题三模型的求解(sections/q3\_solution)}

我们利用模拟退火算法求解上述单目标优化模型。
使用模拟退火算法求解“有效遮蔽时间单目标优化模型”的核心思路是:通过模拟物理退火过程的随机搜索与概率接受机制,在决策变量的可行域内寻找使有效遮蔽时间$\Delta t$最大化的最优解。具体步骤如下:

\noindent\textbf{步骤1 初始化参数}
\begin{itemize}
    \item 初始解:在可行域内随机生成$S_0=(\alpha_0, v_{\text{FY1},0})$,其中$\alpha_0 \in [0,2\pi]$,$v_{\text{FY1},0} \in [70,140]$;
    \item 算法参数:初始温度$T_0=100$,降温系数$k=0.95$,终止温度$T_{\text{end}}=10^{-5}$,每轮迭代次数$L=50$。
\end{itemize}

\noindent\textbf{步骤2 目标函数$\Delta t$计算(核心)}

对任意解$S=(\alpha, v_{\text{FY1}})$,按以下流程计算$\Delta t$:
\begin{itemize}
    \item 依据“无人机位置公式”,计算烟幕投放时刻$t_{\text{FYj,11}}$的无人机坐标;
    \item 依据“烟幕弹起爆位置公式”(含重力下落项),计算起爆时刻$t_{\text{FYj,12}}$的位置;
    \item 依据“烟幕云团运动公式”,计算$t \in [t_{\text{FYj,12}}, t_{\text{FYj,12}}+\Delta t_0]$内的云团中心,结合球面方程确定有效区域;
    \item 在真目标圆柱面($x_1^2+(y_1-y_0)^2=r_0^2, z_1 \in [0,h_0]$)上采样关键点位,通过$\sum_{j=1}^3 a_1^j$判定各时刻遮挡状态,累加有效时长得$\Delta t$。
\end{itemize}

\noindent\textbf{步骤3 邻域解生成}

对当前解$S=(\alpha, v_{\text{FY1}})$添加随机扰动:
\begin{itemize}
    \item 方向角:$\alpha' = \alpha \pm 0.1$(弧度),超出$[0,2\pi]$则取模调整;
    \item 速度:$v_{\text{FY1}}' = v_{\text{FY1}} \pm 5$($\text{m/s}$),超出$[70,140]$则截断至边界。
\end{itemize}

\noindent\textbf{步骤4  判断准则(接受新解)}

计算目标函数差值$\Delta E = \Delta t(S') - \Delta t(S)$:
\begin{itemize}
    \item 若$\Delta E > 0$:直接接受$S'$为当前解;
    \item 若$\Delta E \leq 0$:以概率$P = \exp(\Delta E/T)$接受$S'$($T$为当前温度,温度越高接受概率越大)。
\end{itemize}

\noindent\textbf{步骤5 降温迭代与终止}

每完成$L$次迭代后,按$T = k \cdot T$降温;重复“邻域搜索→接受准则→降温”,直至$T \leq T_{\text{end}}$,输出最优解$S^*=(\alpha^*, v_{\text{FY1}}^*)$及最大$\Delta t^*$。

\begin{table}[H]
\caption{第三问答案}
\label{tab:001} 
\centering
\begin{small}
\begin{tabular}{ccccc}
\toprule[1.5pt]
无人机运动方向 & 无人机运动速度 & 烟幕干扰弹编号 & 烟幕干扰弹投放点的x坐标& 烟幕干扰弹投放点的y坐标 \\
\midrule[1pt]
  3.1356           &140.00                  & 1     & 17653.00                   & 0.88     \\            
  3.1356           &140.00                  & 2     & 17294.61                   & 3.03      \\           
  3.1356           &140.00                  & 3     & 17112.61                   & 4.12      \\           
\bottomrule[1.5pt]
\end{tabular}
\end{small}
\end{table}


\begin{table}[H]
\caption{第三问答案}
\label{tab:031} 
\centering
\begin{small}
\begin{tabular}{ccccc}
\toprule[1.5pt]
    干扰弹投放点的z坐标 &干扰弹起爆点的x坐标&干扰弹起爆点的y坐标&干扰弹起爆点的z坐标&有效干扰时长\\
\midrule[1pt]
1800.00             &17053.81                   & 4.47    &1710.23        & 4.1  \\               
1800.00             &16554.02                   & 7.47    & 1662.88       & 2.7  \\               
1800.00             &16290.83                   & 9.04    & 1631.16       & 1.7  \\                
\bottomrule[1.5pt]
\end{tabular}
\end{small}
\end{table}


\end{document}