\documentclass[../main.tex]{subfiles}
\graphicspath{{figures/}{../figures/}}

\begin{document}
% \todo[color=green!30]{完成问题重述(sections/question\_review)}
\subsection{问题背景}
\par 在当今国际形势下,国防安全至关重要。为应对现代战争中敌方导弹对我方重要目标的攻击,我方会投放烟幕干扰弹影响来袭导弹发现重要目标,从而提升我国的国防实力,对应对外部威胁和维护国家安全具有重要意义。 烟幕干扰弹主要通过在重要目标前方特地区域形成对敌方导弹视线的遮挡从而进行干扰,具有成本低、效费比高等优点。如何通过数学建模通过对无人机的各项参数以及烟雾干扰弹的各项参数进行优化设计,从而实现更为有效的烟雾干扰效果成为了我们有待解决的问题。



\subsection{问题一}
\par 在题目给定的直角坐标系下,将假目标放置在坐标系原点。根据题目给定的无人机$FY1$的位置,飞行速度\text{(}120m/s\text{)},飞行方向,敌方导弹$M1$的位置,受领任务后投放烟幕干扰弹间隔时间\text{(}1.5s\text{)}以及干扰弹起爆间隔时间(3.6s),无人机FY1投放1枚烟幕干扰弹实施对M1的干扰,计算烟幕干扰弹对$M1$的有效遮挡时间。



\subsection{问题二}
\par 已知在题目给定的直角坐标系下,无人机$FY1$的位置以及敌方导弹$M1$的位置已知,无人机$FY1$投放1枚烟幕干扰弹进行干扰.给出无人机$FY1$的飞行方向,无人机的飞行速度,烟幕干扰弹$M1$的投放点,烟幕干扰弹起爆点,使得烟幕干扰弹对来袭导弹的遮蔽时间尽可能长。



\
\subsection{问题四}

\par 本问多加入2架无人机$FY2$和$FY3$,分别投放1枚烟幕干扰弹,实施对敌方导弹的干扰,因此需要重新设计烟幕干扰弹的投放策略。在题目给定的直角坐标系下,3架无人机的位置以及敌方导弹$M1$的位置已知,重新给出3架无人机的飞行方向以及飞行速度、烟幕干扰弹投放点以及起爆点,从而延长对来袭导弹的遮蔽时间。



\subsection{问题五}
\par 已知在题目给定的直角坐标系下,5架无人机的位置以及3枚敌方导弹的位置,每架无人机至多投放3枚烟幕干扰弹.重新给出5架无人机各自的飞行方向以及飞行速度、3枚烟幕干扰弹各自的投放点以及起爆点,从而找到对来袭导弹的遮蔽时间的最优解。






\end{document}