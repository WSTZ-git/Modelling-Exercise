\documentclass[../main.tex]{subfiles}
\graphicspath{{figures/}{../figures/}}

\begin{document}
  % \todo[color=green!30]{完成问题二模型的建立(sections/q2\_build)}
\noindent \textbf{Step 1 目标函数}
\par 在第一问中,我们给出了无人机$FY1$投放的烟幕干扰弹对M1的有效遮蔽时长的$\Delta t$ 的计算公式。在本问中,我们的目标是通过设计无人机$FY1$与其投放的烟幕干扰弹相关参数,使得烟幕干扰弹对导弹$M1$的有效遮蔽时间尽可能长,因此目标函数为:
\begin{align}\label{10.1}
  \max \Delta t
\end{align}



\noindent \textbf{Step 2 决策变量}

\begin{itemize}
\item \textbf{无人机$FY1$的方向}:设$\alpha $为无人机$FY1$与$x$轴正方向的夹角,从而确定无人机$FY1$的飞行方向.
\item \textbf{烟幕干扰弹投放点}:设无人机$FY1$在受领任务 \( t_1 \) s 后投放1枚烟幕干扰弹。
\item  \textbf{无人机$FY1$的飞行速度} :设无人机的飞行速度为 \( v_{FY1} \),无人机受领任务后,保持等高度匀速直线运动。因此在 \( t \) 时刻可以确定无人机的飞行路程 \( S_1 \):
\begin{align}\label{10.2}
  S_1 = v_{FY1} t
\end{align}

\par 该路程所在直线平行于水平面 \( xOy \),且本问已设无人机 \( FY1 \) 与 \( x \) 轴的夹角为 \( \alpha \)。因此可以得到 \( t \) 时刻无人机 \( FY1 \) 的位置坐标:
\begin{align}\label{10.3}
  \begin{cases}
x_{FY1,t} = x_{FY1,0} + v_{FY1} t \cos\alpha \\
y_{FY1,t} = y_{FY1,0} + v_{FY1} t \sin\alpha \\
z_{FY1,t} = z_{FY1,0}
\end{cases}
\end{align}
\item \textbf{烟幕干扰弹起爆点}:设无人机 FY1 投放的烟幕干扰弹在无人机受领任务 \( t_2 \) s 后起爆,因此由问题1可得烟幕干扰弹在投放后到起爆前的位置坐标满足:
\begin{align}\label{10.4}
    \begin{cases}
x_{FY11,t}=x_{FY1,t_1}-v_{FY1}\left( t-t_1 \right)\\
	y_{FY11,t}=y_{FY1,t_1}\\
	z_{FY11,t}=z_{FY1,t_1}-\frac{g\left( t-t_1 \right) ^2}{2}\\
\end{cases}
\quad t_1 \leq t \leq t_2
\end{align}
\par 进而得到烟幕干扰弹起爆时形成的云团中心坐标$\left( x_{FY11,t_2},y_{FY11,t_2},z_{FY11,t_2} \right)$,代入问题1中,得到 $t$ 时刻烟幕云团是否对目标进行遮挡的判断条件$\Delta \left( x_1,y_1,z_1 \right) $.将真目标所在圆柱侧面上任一点坐标$(x_1, y_1, z_1)$ 带入,当 $\Delta \geq 0$ 时,烟幕云团对目标进行遮挡,当 $\Delta < 0$ 时,烟幕云团未对目标形成遮挡,即:
\begin{align}
\begin{cases}
\Delta \geq 0 & \text{遮挡} \\
\Delta < 0 & \text{未遮挡}
\end{cases}
\end{align}\label{10.7}
\end{itemize}

\textbf{Step 3 约束条件}
\begin{itemize}
\item \textbf{无人机的飞行速度}:由于无人机受领任务后,可根据需要瞬时调整飞行方向,然后以70-140m/s的速度等高度匀速直线飞行。因此:
\begin{align}\label{10.8}
  70 \leq v_{\text{FY1}} \leq 140
\end{align}
\end{itemize}


\textbf{Step 4 优化模型}
\par 综上所述,有效遮蔽时间单一目标优化模型为:
\begin{align}
  \begin{array}{c}
	\max \Delta t
  \\
\left\{ \begin{array}{l}
	t\text{时刻无人机的位置坐标:}\\
	\left\{ \begin{array}{l}
	x_{FY1,t}=x_{FY1,0}+v_{FY1}t\cos \alpha \\
	y_{FY1,t}=y_{FY1,0}+v_{FY1}t\sin \alpha \\
	z_{FY1,t}=z_{FY1,0}\\
\end{array} \right.\\
	0\leq \alpha \leq 2\pi \\
  70 \leq v_{\text{FY1}} \leq 140\\
	\text{烟幕导弹起爆时的位置坐标:}\\
	\left\{ \begin{array}{l}
	x_{FY11,t_2}=x_{FY1,t_1}-v_{FY1}\left( t_2-t_1 \right) \cos \lambda\\
	y_{FY11,t_2}=y_{FY1,t_1}-v_{FY1}\left( t_2-t_1 \right) \sin \lambda\\
	z_{FY11,t_2}=z_{FY1,t_1}-\frac{g\left( t_2-t_1 \right) ^2}{2}\\
\end{array} \right.\\
	t\text{时刻云团中心坐标:}\\
	\left\{ \begin{array}{l}
	x_{FY11,t}=x_{FY11,t_2}\\
	y_{FY11,t}=y_{FY11,t_2}\\
	z_{FY11,t}=z_{FY11,t_2}-v_1\left( t-t_2 \right)\\
	t_2\leq t\leq t_2+\Delta t_0\\
\end{array} \right.\\
	\text{烟雾有效遮挡范围的球形区域:}\\
	O_{FY11,t}:\left( x-x_{FY11,t} \right) ^2+\left( y-y_{FY11,t} \right) ^2+\left( z-z_{FY11,t} \right) ^2=r^2\left( 2.5 \right)\\
	\text{圆柱面上点坐标:}\\
	\left\{ \begin{array}{c}
	\begin{array}{l}
	x_{1}^{2}+\left( y_{1}^{2}-y_0 \right) ^2=r_{0}^{2}\\
	z_1\in \left[ 0,h_0 \right]\\
\end{array}\\
\end{array} \right.\\
	\text{烟幕云团遮挡判断条件:}\\
	\left\{ \begin{matrix}
	\Delta \geq 0&		\text{遮挡}\\
	\Delta <0&		\text{未遮挡}\\
\end{matrix} \right.\\
\end{array} \right.  
\end{array}
\end{align}










  \end{document}