% !Mode:: "TeX:UTF-8"
% !BIB program = bibtex
% !TEX program = xelatex

% \documentclass[withoutpreface,bwprint,ebook]{../../Template/cumcmthesis} % 普通草稿
% \documentclass[draft,ebook]{../../Template/cumcmthesis} % 电子书草稿
\documentclass{../../Template/cumcmthesis} % 带封面的版本
% \documentclass[withoutpreface,bwprint]{../../Template/cumcmthesis} %去掉封面与编号页

% 基本信息
\title{} % 文章标题写到这里
\tihao{}
\baominghao{}
\schoolname{}
\membera{} % 成员
\memberb{}
\memberc{}
\supervisor{}
\yearinput{}
\monthinput{}
\dayinput{}

% 载入路径
\makeatletter
\def\input@path{{sections/}{figures/}{../figures/}}
\makeatother

% 导入符号
% 符号定义文件,命令所有文件共享
% 并在 notations 打印

\glsxtrnewsymbol[
	description={%
		an example symbol % 对符号的描述
	},
	unit={\si{m^2}} % 单位,不显示可以不写
]{e}{\ensuremath{\mathcal{E}}}
% 第一个参数e是后期可以直接\gls{e}代表之后的符号


\usepackage{../../Styles/mystyle-cumcmthesis}
\usepackage{subfiles}  % Best loaded last in the preamble
\begin{document}

% TODO 提交最终版论文时别忘了选择 7,8行版本
\listoftodos
\maketitle

\subfile{sections/abstract}

\section{问题重述}
  \subfile{sections/question_review}
\section{问题的分析}
  \subfile{sections/analyse}
\section{模型的假设}
  \subfile{sections/assumptions}
\section{符号说明}
  \subfile{sections/notations}
\section{模型的建立与求解}

\subsection{问题一}
  \subsubsection{问题一的模型建立}
    \subfile{sections/q1_build}
  \subsubsection{问题一的模型求解}
    \subfile{sections/q1_solution}
\subsection{问题二}
  \subsubsection{模型建立}
    \subfile{sections/q2_build}
  \subsubsection{模型求解}
    \subfile{sections/q2_solution}
\subsection{问题三}
  \subsubsection{模型建立}
    \subfile{sections/q3_build}
  \subsubsection{模型求解}
    \subfile{sections/q3_solution}
\subsection{问题四}
\subsubsection{模型建立}
\subfile{sections/q4_build}
\subsubsection{模型求解}
\subfile{sections/q4_solution}

\section{模型的评价}
  \subfile{sections/model_review}

% 参考文献部分

\newpage
\nocite{*}
\bibliography{../reference/reference}


\newpage
\begin{appendices} % 附录
  \section{支撑材料目录与代码环境依赖}
    本文支撑材料目录结构如下
    \lstinputlisting{sections/tree.txt}

  \section{导入其他代码的测试}
  \lstinputlisting[language=python]{../src/test.py}

\end{appendices}

\end{document}
