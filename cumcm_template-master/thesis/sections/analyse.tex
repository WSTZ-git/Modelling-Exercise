\documentclass[../main.tex]{subfiles}
\graphicspath{{figures/}{../figures/}}

\begin{document}
  \todo[color=green!30]{完成问题分析(sections/analyse)}
  \subsection{问题一的分析}
  问题一已知盘入螺线的螺距,龙头前把手行进速度以及初始位置,求所有把手的位置与速度。首先,我们构建符合该等距螺线的极坐标系,求出盘入螺线的极坐标方程与直角坐标方程。通过曲线积分的形式,得到龙头前把手在某一时间点与其极角之间的关系,从而确定龙头前把手在不同时刻的位置。并通过前把手中心与后把手中心之间的距离,建立距离与极角之间的关系式,确定后把手在各时刻的位置。最后,通过弧微分等计算公式,得到相邻两个后把手中心速度的关系式,并通过迭代计算,得到各把手的速度。
  \subsection{问题二的分析}
  问题二要求得到舞龙队盘入发生碰撞前的终止时刻,以及该时刻各把手的位置与速度。首先,我们利用反证法证明板凳龙首次发生碰撞一定是第1节板凳或第2节板凳与其他板凳碰撞,缩小碰撞判断范围。接着,我们表示出各节板凳顶点的直角坐标以划定边界,便于碰撞检测。然后,我们定义何为发生碰撞,认为任意两节板凳对应的边界有交点则判断碰撞。据此判断条件,估计碰撞时间范围,对估计区间进行变长步搜索,得到精确的碰撞时间。最后,将碰撞时间代入问题一建立的模型方法计算该时刻所有把手的位置与速度。
  \subsection{问题三的分析}
  问题三已知调头空间直径,要求得到龙头前把手盘入掉头空间的最小螺距。首先,通过问题二得到的计算结果,得到特定螺距下舞龙队达到终止时刻(即不发生碰撞的最晚时刻)时龙头前把手的极径,并与调头空间半径进行比较来确定螺距上限。然后设定初始值,并根据初始值设定初始步长,按步长减小螺距,计算龙头到达调头边界的时间,并通过问题二的方法判断在这个时间段内,板凳之间是否会发生碰撞。若发生碰撞则确定螺线范围,退回上一步,通过减少步长并再次循环,逐步逼近螺距的最小值。
  \subsection{问题四的分析}
  问题四已知盘入螺线的螺距,以及调头空间的大致几何图形,要求判断能否通过调整圆弧使得调头曲线过短,以及求解各个时间点,舞龙队的位置与速度。首先,通过分析盘入、盘出螺线以及调头圆弧之间的几何关系,利用角度关系、斜率公式和圆弧长度公式,推导得出调头曲线长度与前后两段调头圆弧半径比例无关。然后计算调头曲线长度,以及计算出调头曲线几个关键点的坐标。接着对龙头前把手和其他把手分别位于盘入螺线、调头曲线、盘出螺线等四种情形进行讨论。在每种情形中,引入判断角,借助几何关系建立方程,从而确定不同把手在各个时刻的位置。最后,将它们的位置代入问题一的模型中,求得不同把手在各时刻的速度。
  \subsection{问题五的分析}
  问题五时在问题四的基础上,给定龙头各把手速度的约束条件,求解龙头的最大前进速度。由于板凳龙各把手的位置已知,因此根据问题一构建的模型中的速度迭代公式,可以构造以龙头速度为自变量,各把手速度为因变量的约束模型。从而求得龙头前进速度的范围,取范围的最大值,即可求得龙头的最大前进速度。
  
\end{document}