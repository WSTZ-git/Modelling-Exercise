\documentclass[lang=cn,newtx,10pt,scheme=chinese]{../Template/elegantbook}

\title{Intial Draft}
\subtitle{空}

\author{邹文杰}
\institute{无}
\date{2024/10/25}
\version{ElegantBook-4.5}
\bioinfo{自定义}{信息}

\extrainfo{宠辱不惊,闲看庭前花开花落;\\去留无意,漫随天外云卷云舒。}

\setcounter{tocdepth}{3}

\logo{logo-blue.png}
\cover{cover.png}

% 本文档额外使用的宏包和命令
\usepackage{../Styles/mystyle-elegantbook}

\begin{document}

\maketitle

\frontmatter

\tableofcontents

\mainmatter


\chapter{问题1}

\section{螺线方程}

在题目图4中的直角坐标系下,以坐标原点\(O\)为极点建立极坐标系,设图4中的等距螺线\(\varGamma\)的极坐标方程为
\begin{align}\label{1..1}
\varGamma :\rho =a+b\theta.
\end{align}
其中\(\rho\)为极径,\(\theta\)为极角,\(a\),\(b\)均为待定常数.设龙头前把手中心的初始位置\(P_0\)极坐标和直角坐标分别为\((\rho _0,\theta _0)\),\((x_0,y_0)\),记\(d_0(\mathrm{m})\)为图4中等距螺线的螺距,则由题可知
\begin{align}
d_0 = 0.55,\rho _0 = 16d_0 = 8.8,\theta _0 = 16\times 2\pi = 32\pi.
\end{align}
又由图4可知图中等距螺线\(\varGamma\)过原点\(O\)和\((\rho _0,\theta _0)\)点,于是将\((0,0)\)和\((\rho _0,\theta _0)\)代入\eqref{1..1}式解得
\begin{align}
    a = 0,b = \frac{d_0}{2\pi}=\frac{0.55}{2\pi}。
\end{align}
因此,等距螺线\(\varGamma\)的极坐标方程为
\begin{align}
\varGamma :\rho =\frac{d_0}{2\pi}\theta ,(0\leqslant \theta \leqslant 32\pi) .\label{0.1}
\end{align}
再利用
\[\begin{cases}
x=\rho \cos \theta\\
y=\rho \sin \theta\\
\end{cases},\]
得到等距螺线\(\varGamma\)的直角坐标方程为
\begin{align}
\begin{cases}
x=\frac{d_0}{2\pi}\theta \cos \theta\\
y=\frac{d_0}{2\pi}\theta \sin \theta\\
\end{cases},(0\leqslant \theta \leqslant 32\pi) .\label{0.0}
\end{align}

\section{计算龙头前把手中心在各时刻的位置}\label{1.(2)}

设龙头前把手的行进速度为\(v_0\),由题可知\(v_0 = 1(\mathrm{m}\cdot s^{-1})\).记龙头前把手中心在第\(t\)秒时的位置为\(P_0(t)\),\(P_0(t)\)点的极坐标和直角坐标分别为\((\rho _0(t),\theta _0(t))\)和\((x_0(t),y_0(t))\),
则利用第一型曲线积分计算公式可得,曲线\(\overgroup{P_0P_0(t)}\)的长度\(S\)为
\begin{align}
S=\int_{\overgroup{P_0P_0(t)}}{\mathrm{d}s}=\int_{\theta _0(t)}^{\theta _0}{\sqrt{[\rho (\theta )]^2+[\rho ' (\theta )]^2}\mathrm{d}\theta}.\label{1.1}
\end{align}
又由题可知
\begin{align}
S = v_0t.\label{1.2}
\end{align}
联立\eqref{0.1}\eqref{1.1}\eqref{1.2}式解得
\begin{align}
\theta _t\sqrt{\theta _{t}^{2}+1}-\ln (\theta _0(t)+\sqrt{(\theta _0(t))^2+1}) =\theta _0\sqrt{\theta _{0}^{2}+1}+\ln (\theta _0+\sqrt{\theta _{0}^{2}+1}) -\frac{4\pi}{d_0}v_0t .
\end{align}
根据上式,利用Python求解得到当\(t\in \{ 1,2,\cdots ,300 \}\)时,\(P_0(t)\)的极角\(\theta _0(t)\),再将其代入\eqref{0.1}式得到,\(P_0(t)\)的极坐标\((\rho _0(t),\theta _0(t))\)见表1.再利用
\begin{align}
\begin{cases}
x_0(t)=\rho _0(t)\cos \theta _0(t)\\
y_0(t)=\rho _0(t)\sin \theta _0(t)\\
\end{cases},
\end{align}
得到当\(t\in \{ 1,2,\cdots ,300 \}\)时,\(P_0(t)\)的的直角坐标\((x_0(t),y_0(t))\)见表2.

\section{计算板凳龙其余各节板凳的后把手中心在各时刻的位置}

记第\(i\)节板凳的后把手中心在第\(t\)秒时的位置为\(P_{i}(t)\),并设\(P_{i}(t)\)的极坐标和直角坐标分别为\((\rho _{i}(t),\theta _{i}(t))\)和\((x_{i}(t),y_{i}(t))\),再记\(P_{i-1}(t)\)与\(P_{i}(t)\)之间的距离为\(| P_{i-1}(t)P_{i}(t)|(\mathrm{m}) (0\leqslant i\leqslant 223)\).再记第$i$节板凳前把手中心与后把手中心之间的距离为\(l_i(\mathrm{m})(1\leq i\leq 223)\),则由条件可知
\begin{gather}
| P_0(t)P_1(t)| = l_1 = 3.41 - 2\times 0.275 = 2.86;
\\
| P_{i-1}(t)P_{i}(t)| = l_i = 2.2 - 2\times 0.275 = 1.65,2\leqslant i\leqslant 223.
\end{gather}

\subsection{计算第1节板凳的后把手中心在第\(t\)秒时的位置\(P_1(t)\)}\label{2.(1)}

当\(t\in \{ 1,2,\cdots ,300 \}\)时,由极坐标系的两点之间距离公式可得
\begin{align}
l_{1}^{2}=| P_0(t)P_1(t)|^2=(\rho _1(t))^2+(\rho _0(t))^2 - 2\rho _1(t)\rho _0(t)\cos (\theta _0(t)-\theta _1(t)) .\label{2.1} 
\end{align}
又因为板凳龙各把手中心均位于螺线\(\varGamma\)上,所以再结合\eqref{0.0}式可得
\begin{align}
\rho _1(t)=\frac{d_0}{2\pi}\theta _1(t),(0\leqslant \theta _1(t)\leqslant 32\pi) .\label{2.2}
\end{align}
因此联立\eqref{2.1}\eqref{2.2}式可得
\begin{align}
l_{1}^{2}=\frac{d_{0}^{2}}{4\pi ^2}[(\theta _1(t))^2+(\theta _0(t))^2 - 2\theta _1(t)\theta _0(t)\cos (\theta _0(t)-\theta _1(t))] .\label{2.0}
\end{align}
根据上式利用Python求解\(\theta _1(t)\),可能得到多个不同解.不妨设这些为不同的解为\(\alpha _{j}^{1}(t) (j = 1,2,\cdots ,m)\),注意到一定有\(\theta _1(t)>\theta _0(t)\),因此令
\begin{align}
A_1 = \{ \alpha _{j}^{1}(t) |\alpha _{j}^{1}(t) >\theta _0(t),j = 1,2,\cdots ,m \} .
\end{align}
又因为第1个把手的极角一定是$A_1$中最小的,所以
\begin{align}
\theta _1(t)=\underset{\alpha _j(t)\in A}{\min}\,\alpha _{j}^{1}(t).
\end{align}
再将上述求得的\(\theta _1(t)\)代入\eqref{2.2}式就能得到此时\(P_1(t)\)的极坐标\((\rho _1(t),\theta _1(t))\).令\(t\)依次取\(1\),\(2\),\(\cdots\),\(300\),反复进行上述操作就能得到,当\(t\in \{ 1,2,\cdots ,300 \}\)时,\(P_1(t)\)的极坐标\((\rho _1(t),\theta _1(t))\).再利用
\begin{align}
\begin{cases}
x_1(t)=\rho _1(t)\cos \theta _1(t)\\
y_1(t)=\rho _1(t)\sin \theta _1(t)\\
\end{cases},    
\end{align}
得到当\(t\in \{ 1,2,\cdots ,300 \}\)时,\(P_1(t)\)的直角坐标\((x_1(t),y_1(t))\).

\subsection{计算第\(i(2\leqslant i\leqslant 223)\)节板凳的后把手中心在第\(t\)秒时的位置\(P_{i}(t)\)}\label{2.(2)}

当\(i\in \{ 2,3,\cdots ,223 \}\)时,由\eqref{2.(1)}同理可得,当\(t\in \{ 1,2,\cdots ,300 \}\)时,我们有
\begin{gather}
l_{i}^{2}=| P_{i-1}(t)P_{i}(t)|^2=(\rho _{i}(t))^2+(\rho _{i-1}(t))^2 - 2\rho _{i}(t)\rho _{i-1}(t)\cos (\theta _{i-1}(t)-\theta _{i}(t)) .\label{2.2.1}
\\
\rho _{i}(t)=\frac{d_0}{2\pi}\theta _{i}(t),(0\leqslant \theta _{i}(t)\leqslant 32\pi) .\label{2.2.2}
\end{gather}
从而联立\eqref{2.2.1}\eqref{2.2.2}式可得
\begin{align}
l_{i}^{2}=\frac{d_{0}^{2}}{4\pi ^2}[(\theta _{i}(t))^2+(\theta _{i-1}(t))^2 - 2\theta _{i}(t)\theta _{i-1}(t)\cos (\theta _{i-1}(t)-\theta _{i}(t))] .\label{2.0.0}
\end{align}
根据上式利用Python求解\(\theta _{i}(t)\),可能得到多个不同解.不妨设这些为不同的解为\(\alpha _{j}^{i}(t) (j = 1,2,\cdots ,m)\),注意到一定有\(\theta _{i}(t)>\theta _{i-1}(t)\),因此令
\begin{align}
A_i = \{ \alpha _{j}^{i}(t) |\alpha _{j}^{i}(t) >\theta _{i-1}(t),j = 1,2,\cdots ,m \},
\end{align}
又因为第\(i - 1\)个把手的极角一定是$A_i$中最小的,所以
\begin{align}
\theta _i(t)=\underset{\alpha _{j}^{i}(t)\in A_i}{\min}\alpha _{j}^{i}(t).   
\end{align}
再将上述求得的\(\theta _i(t)\)代入\eqref{2.2}式就能得到此时\(P_{i}(t)\)的极坐标\((\rho _{i}(t),\theta _{i}(t))\).令\(t\)依次取\(1\),\(2\),\(\cdots\),\(300\),反复进行上述操作就能得到,当\(t\in \{ 1,2,\cdots ,300 \}\)时,\(P_{i}(t)\)的极坐标\((\rho _{i}(t),\theta _{i}(t))\).再利用
\begin{align}
\begin{cases}
x_{i}(t)=\rho _{i}(t)\cos \theta _{i}(t)\\
y_{i}(t)=\rho _{i}(t)\sin \theta _{i}(t)\\
\end{cases}, 
\end{align}
得到当\(t\in \{ 1,2,\cdots ,300 \}\)时,\(P_{i}(t)\)的直角坐标\((x_{i}(t),y_{i}(t))\).

综上所述,令\(i\)依次取\(1\),\(2\),\(\cdots\),\(223\),按照上述\eqref{2.(1)}\eqref{2.(2)}的方式,利用Python不断迭代计算就能得到每秒板凳龙各把手中心的位置直角坐标见表3.

\section{计算板凳龙其余各节板凳的后把手中心在各时刻的速度}

记第\(i(1\leqslant i\leqslant 223)\)节板凳的后把手中心在第\(t\)秒时的速度为\(v_i(t)\).根据\eqref{1.(2)},\eqref{2.(1)},\eqref{2.(2)}得到的第\(i(1\leqslant i\leqslant 223)\)节板凳的后把手中心第\(t\)秒时的位置\(P_{i}(t)\)的极坐标\((\rho _{i}(t),\theta _{i}(t))\),于是当\(i\in \{ 1,2,\cdots ,223 \}\)时,对\eqref{2.0}\eqref{2.0.0}式两边同时对\(t\)求导可得
\begin{align}
\frac{\mathrm{d}\theta _i}{\mathrm{d}t}=\frac{\theta _{i - 1}+\theta _i\cos(\theta _{i - 1}-\theta _i)-\theta _i\theta _{i - 1}\sin(\theta _{i - 1}-\theta _i)}{\theta _i+\theta _i\theta _{i - 1}\sin(\theta _{i - 1}-\theta _i)-\theta _{i - 1}\cos(\theta _{i - 1}-\theta _i)}\cdot \frac{\mathrm{d}\theta _{i - 1}}{\mathrm{d}t}.\label{3.0}
\end{align}
设第\(i(1\leqslant i\leqslant 223)\)节板凳的后把手中心在充分短的时间\(\mathrm{d}t\)内经过的路程微分为\(\mathrm{d}s_i\),又因为各把手中心始终在螺线\(\varGamma\)上,从而各把手的路程微分\(\mathrm{d}s_i\)就是螺线\(\varGamma\)的弧微分,所以利用\eqref{0.1}式及弧微分的计算公式可得
\begin{align}
\mathrm{d}s_i=\sqrt{[\rho (\theta _i)]^2+[\rho \prime (\theta _i)]^2}\mathrm{d}\theta _i=\frac{d_0}{2\pi}\sqrt{{\theta _i}^2+1}\mathrm{d}\theta _i,i\in \{ 1,2,\cdots ,223 \}  
\end{align}
故当\(t\in \{ 1,2,\cdots ,300 \}\)时,由瞬时速度的定义可得
\begin{align}
v_i(t) =\frac{\mathrm{d}s_i}{\mathrm{d}t}=\frac{d_0}{2\pi}\frac{\sqrt{{\theta _i}^2 + 1}\mathrm{d}\theta _i}{\mathrm{d}t}, i\in \{1, 2, \cdots, 223\}. \label{3.1}
\end{align}
联立\eqref{3.0}\eqref{3.1}式得到
\begin{align}
|v_i(t)| &= \left|\frac{d_0}{2\pi}\frac{\sqrt{{\theta _i}^2 + 1}\mathrm{d}\theta _i}{\mathrm{d}t}\right| 
= \frac{|\theta _{i - 1} + \theta _i\cos(\theta _{i - 1} - \theta _i) - \theta _i\theta _{i - 1}\sin(\theta _{i - 1} - \theta _i)|}{|\theta _i + \theta _i\theta _{i - 1}\sin(\theta _{i - 1} - \theta _i) - \theta _{i - 1}\cos(\theta _{i - 1} - \theta _i)|}\sqrt{\frac{1 + {\theta _i}^2}{1 + {\theta _{i - 1}}^2}}\left|\frac{\mathrm{d}\theta _{i - 1}}{\mathrm{d}t}\right| \\
&= \frac{|\theta _{i - 1}(t) + \theta _i\cos(\theta _{i - 1} - \theta _i) - \theta _i\theta _{i - 1}\sin(\theta _{i - 1} - \theta _i)|}{|\theta _i + \theta _i\theta _{i - 1}\sin(\theta _{i - 1} - \theta _i) - \theta _{i - 1}\cos(\theta _{i - 1} - \theta _i)|}\sqrt{\frac{1 + {\theta _i}^2}{1 + {\theta _{i - 1}}^2}}|v_{i - 1}(t)|, i\in \{1, 2, \cdots, 223\}.
\end{align}
其中\(\theta _i = \theta _i(t)\),\(\theta _{i - 1} = \theta _{i - 1}(t)\),\(v_0(t) \equiv 1, \forall t\geqslant 0\).又因为\(v_i(t) (1\leqslant i\leqslant 223)\)均大于\(0\),所以上式可化为
\begin{align}
v_i(t) = \sqrt{\frac{1 + {\theta _i}^2}{1 + {\theta _{i - 1}}^2}}\frac{|\theta _{i - 1} + \theta _i\cos(\theta _{i - 1} - \theta _i) - \theta _i\theta _{i - 1}\sin(\theta _{i - 1} - \theta _i)|}{|\theta _i + \theta _i\theta _{i - 1}\sin(\theta _{i - 1} - \theta _i) - \theta _{i - 1}\cos(\theta _{i - 1} - \theta _i)|}v_{i - 1}(t), i\in \{1, 2, \cdots, 223\},
\end{align}
其中\(\theta _i = \theta _i(t)\),\(\theta _{i - 1} = \theta _{i - 1}(t)\),\(v_0(t) \equiv 1, \forall t\geqslant 0\).
于是根据上式,令\(i\)依次取\(1, 2, \cdots, 223\),再利用Python进行迭代计算,就能得到板凳龙的第\(i(1\leqslant i\leqslant 223)\)节板凳的后把手中心在第\(t\)秒时的速度\(v_i(t)\).
再令\(t\)依次取\(1, 2, \cdots, 300\),反复进行上述操作,就能得到当\(t\in \{1, 2, \cdots, 300\}\)时,板凳龙的第\(i(1\leqslant i\leqslant 223)\)节板凳的后把手中心每秒的速度见表\(7\).



\chapter{问题2}

\section{计算各节板凳四个顶点的坐标}

$\forall t\in \mathbb{N}$,根据\textbf{问题1}求解得到龙头前把手中心、第$i(i=1,2,\cdots,223)$节板凳后把手中心在第$t$秒的直角坐标分别为$(x_0(t),y_0(t))$,$(x_i(t),y_i(t))$.设所有板凳的板宽均为$w$,板凳把手中心离最近的板头距离为$h$,则由条件可知$w=0.3m,h=0.275m$.当第$i(i=1,2,\cdots,223)$节板凳后把手刚盘入螺线(即后把手恰好在初始位置)时,记离原点较远且离$P_i$较近的顶点为$A_i$,再按顺时针方向分别记其余顶点为$B_i,C_i,D_i$.记$A_i,B_i,C_i,D_i$在第$i$秒时的位置分别为$A_i(t),B_i(t),C_i(t),D_i(t)$,其直角坐标分别为$(x_{A_i}(t),y_{A_i}(t)),(x_{B_i}(t),y_{B_i}(t)),(x_{C_i}(t),y_{C_i}(t)),(x_{D_i}(t),y_{D_i}(t))$.于是根据向量垂直坐标变换公式可得,对$\forall i\in {0,1,2,\cdots,222}$,都有
\begin{gather}
\overrightarrow{P_{i+1}\left( t \right) P_i\left( t \right) }=\left( x_i\left( t \right) -x_{i+1}\left( t \right) ,y_i\left( t \right) -y_{i+1}\left( t \right) \right) ,\label{problem-2.1}
\\
\overrightarrow{A_i\left( t \right) D_i\left( t \right) }=\overrightarrow{B_i\left( t \right) C_i\left( t \right) }=\left( -\left[ y_i\left( t \right) -y_{i+1}\left( t \right) \right] ,x_i\left( t \right) -x_{i+1}\left( t \right) \right) ,\label{problem-2.2}
\\
\left| \overrightarrow{A_i\left( t \right) D_i\left( t \right) } \right|=\left| \overrightarrow{B_i\left( t \right) C_i\left( t \right) } \right|=w.\label{problem-2.3}
\end{gather}
再记$B_iC_i$和$A_iD_i$的中点分别为$E_i,F_i$,它们在第$t$秒时的位置分别为$E_i(t),F_i(t)$,直角坐标分别为
\begin{align}
E_i(t)=(x_{E_i}(t),y_{E_i}(t)),
\\
F_i(t)=(x_{F_i}(t),y_{F_i}(t)).
\end{align}
从而
\begin{gather}
\overrightarrow{P_{i+1}\left( t \right) E_i\left( t \right) }=\left( x_{E_i}\left( t \right) -x_{i+1}\left( t \right) ,y_{E_i}\left( t \right) -x_{i+1}\left( t \right) \right) ,\label{problem-2.4}
\\
\overrightarrow{P_i\left( t \right) F_i\left( t \right) }=\left( x_{F_i}\left( t \right) -x_i\left( t \right) ,y_{F_i}\left( t \right) -x_i\left( t \right) \right) ,\label{problem-2.5}
\\
\left| \overrightarrow{P_{i+1}\left( t \right) E_i\left( t \right) } \right|=\left| \overrightarrow{P_i\left( t \right) F_i\left( t \right) } \right|=h.\label{problem-2.6}
\\
\overrightarrow{E_i\left( t \right) B_i\left( t \right) }=\left( x_{B_i}\left( t \right) -x_{E_i}\left( t \right) ,y_{B_i}\left( t \right) -y_{E_i}\left( t \right) \right) ,\label{problem-2..1}
\\
\overrightarrow{E_i\left( t \right) C_i\left( t \right) }=\left( x_{C_i}\left( t \right) -x_{E_i}\left( t \right) ,y_{C_i}\left( t \right) -y_{E_i}\left( t \right) \right) ,\label{problem-2..2}
\\
\overrightarrow{F_i\left( t \right) A_i\left( t \right) }=\left( x_{A_i}\left( t \right) -x_{F_i}\left( t \right) ,y_{A_i}\left( t \right) -y_{F_i}\left( t \right) \right) ,\label{problem-2..3}
\\
\overrightarrow{F_i\left( t \right) D_i\left( t \right) }=\left( x_{D_i}\left( t \right) -x_{F_i}\left( t \right) ,y_{D_i}\left( t \right) -y_{F_i}\left( t \right) \right) .\label{problem-2..4}
\end{gather}
由$E_i,P_i,P_{i+1},F_i$共线可得
\begin{gather}
\overrightarrow{P_{i+1}\left( t \right) E_i\left( t \right) }=-\frac{\overrightarrow{P_{i+1}\left( t \right) P_i\left( t \right) }}{\left| \overrightarrow{P_{i+1}\left( t \right) P_i\left( t \right) } \right|}\cdot \left| \overrightarrow{P_{i+1}\left( t \right) E_i\left( t \right) } \right|,\label{problem-2.7}
\\
\overrightarrow{P_i\left( t \right) F_i\left( t \right) }=-\frac{\overrightarrow{P_{i+1}\left( t \right) P_i\left( t \right) }}{\left| \overrightarrow{P_{i+1}\left( t \right) P_i\left( t \right) } \right|}\cdot \left| \overrightarrow{P_{i+1}\left( t \right) E_i\left( t \right) } \right|.\label{problem-2.8}
\end{gather}
联立\eqref{problem-2.1}\eqref{problem-2.4}\eqref{problem-2.5}\eqref{problem-2.6}\eqref{problem-2.7}\eqref{problem-2.8}式可得
\begin{gather}
\begin{cases}
x_{E_i}\left( t \right) =x_{i+1}\left( t \right) -\frac{h}{l_{i+1}}\left( x_i\left( t \right) -x_{i+1}\left( t \right) \right)\\
y_{E_i}\left( t \right) =y_{i+1}\left( t \right) -\frac{h}{l_{i+1}}\left( y_i\left( t \right) -y_{i+1}\left( t \right) \right)\\
\end{cases},\label{problem-2.9}
\\
\begin{cases}
x_{F_i}\left( t \right) =x_i\left( t \right) +\frac{h}{l_{i+1}}\left( x_i\left( t \right) -x_{i+1}\left( t \right) \right)\\
y_{F_i}\left( t \right) =y_i\left( t \right) +\frac{h}{l_{i+1}}\left( y_i\left( t \right) -y_{i+1}\left( t \right) \right)\\
\end{cases}.\label{problem-2.10}
\end{gather}
又由$E_i$是$B_i,C_{i}$的中点和$F_i$是$A_i,D_i$的中点可得
\begin{gather}
\overrightarrow{E_i\left( t \right) B_i\left( t \right) }=-\frac{\overrightarrow{B_i\left( t \right) C_i\left( t \right) }}{2},\label{problem-2.11}
\\
\overrightarrow{E_i\left( t \right) C_i\left( t \right) }=\frac{\overrightarrow{B_i\left( t \right) C_i\left( t \right) }}{2},\label{problem-2.12}
\\
\overrightarrow{F_i\left( t \right) A_i\left( t \right) }=-\frac{\overrightarrow{A_i\left( t \right) D_i\left( t \right) }}{2},\label{problem-2.13}
\\
\overrightarrow{F_i\left( t \right) D_i\left( t \right) }=\frac{\overrightarrow{A_i\left( t \right) D_i\left( t \right) }}{2}.   \label{problem-2.14} 
\end{gather}
因此联立\eqref{problem-2.2}\eqref{problem-2.3}\eqref{problem-2..1}\eqref{problem-2..2}\eqref{problem-2..3}\eqref{problem-2..4}\eqref{problem-2.9}\eqref{problem-2.10}\eqref{problem-2.11}\eqref{problem-2.12}\eqref{problem-2.13}式,解得
\begin{gather}
\begin{cases}
x_{A_i}\left( t \right) =x_i\left( t \right) +\frac{h}{l_{i+1}}\left( x_i\left( t \right) -x_{i+1}\left( t \right) \right) +\frac{y_i\left( t \right) -y_{i+1}\left( t \right)}{2}\\
y_{A_i}\left( t \right) =y_i\left( t \right) +\frac{h}{l_{i+1}}\left( y_i\left( t \right) -y_{i+1}\left( t \right) \right) -\frac{x_i\left( t \right) -x_{i+1}\left( t \right)}{2}\\
\end{cases},
\\
\begin{cases}
x_{B_i}\left( t \right) =x_{i+1}\left( t \right) -\frac{h}{l_{i+1}}\left( x_i\left( t \right) -x_{i+1}\left( t \right) \right) +\frac{y_i\left( t \right) -y_{i+1}\left( t \right)}{2}\\
y_{B_i}\left( t \right) =y_{i+1}\left( t \right) -\frac{h}{l_{i+1}}\left( y_i\left( t \right) -y_{i+1}\left( t \right) \right) -\frac{x_i\left( t \right) -x_{i+1}\left( t \right)}{2}\\
\end{cases},
\\
\begin{cases}
x_{C_i}\left( t \right) =x_{i+1}\left( t \right) -\frac{h}{l_{i+1}}\left( x_i\left( t \right) -x_{i+1}\left( t \right) \right) -\frac{y_i\left( t \right) -y_{i+1}\left( t \right)}{2}\\
y_{C_i}\left( t \right) =y_{i+1}\left( t \right) -\frac{h}{l_{i+1}}\left( y_i\left( t \right) -y_{i+1}\left( t \right) \right) +\frac{x_i\left( t \right) -x_{i+1}\left( t \right)}{2}\\
\end{cases},
\\
\begin{cases}
x_{D_i}\left( t \right) =x_i\left( t \right) +\frac{h}{l_{i+1}}\left( x_i\left( t \right) -x_{i+1}\left( t \right) \right) -\frac{y_i\left( t \right) -y_{i+1}\left( t \right)}{2}\\
y_{D_i}\left( t \right) =y_i\left( t \right) +\frac{h}{l_{i+1}}\left( y_i\left( t \right) -y_{i+1}\left( t \right) \right) +\frac{x_i\left( t \right) -x_{i+1}\left( t \right)}{2}\\
\end{cases}.
\end{gather}

\section{计算板凳龙发生碰撞的时间}

对$\forall t>0$,板凳龙在第$t$秒发生碰撞的充要条件是:当$t=t_0$时,存在$s,k\in \{1,2,\cdots,223\}$,使得$A_s(t),$ $B_s(t),$ $C_s(t),$ $D_s(t)$和$A_k(t)$,$B_k(t)$,$C_k(t)$,$D_k(t)$所对应的两节板凳在第$t$秒发生碰撞,即此时矩形$A_s(t)B_s(t)C_s(t)D_s(t)$和矩形$A_k(t)$ $B_k(t)$ $C_k(t)$ $D_k(t)$有交点.于是我们只需要判断在第$t$秒时,对$\forall i,j\in{1,2,\cdots,223}$,矩形$A_i(t)$ $B_i(t)$ $C_i(t)$ $D_i(t)$和矩形$A_j(t)$ $B_j(t)$ $C_j(t)$ $D_j(t)$是否有交点即可.

\subsection{判断矩形$A_i(t)B_i(t)C_i(t)D_i(t)$和矩形$A_j(t)B_j(t)C_j(t)D_j(t)$是否有交点的算法}\label{判断矩形相交的算法}

任取矩形$A_i(t)B_i(t)C_i(t)D_i(t)$的一条边记为$X_1(t)Y_1(t)$,矩形$A_j(t)B_j(t)C_j(t)D_j(t)$的一条边记为$X_2(t)Y_2(t)$.
则利用向量叉乘的性质(详见\href{https://zhuanlan.zhihu.com/p/644689588}{几何算法:判断两条线段是否相交})可得
\begin{align}
\begin{cases}
\text{若}\left( \overrightarrow{X_1\left( t \right) Y_1\left( t \right) }\times \overrightarrow{X_1\left( t \right) X_2\left( t \right) } \right) \cdot \left( \overrightarrow{X_1\left( t \right) Y_1\left( t \right) }\times \overrightarrow{X_1\left( t \right) Y_2\left( t \right) } \right) >0,\text{则}\overrightarrow{X_1\left( t \right) Y_1\left( t \right) },\overrightarrow{X_2\left( t \right) Y_2\left( t \right) }\text{相交},\\
\text{若}\left( \overrightarrow{X_1\left( t \right) Y_1\left( t \right) }\times \overrightarrow{X_1\left( t \right) X_2\left( t \right) } \right) \cdot \left( \overrightarrow{X_1\left( t \right) Y_1\left( t \right) }\times \overrightarrow{X_1\left( t \right) Y_2\left( t \right) } \right) \leqslant 0,\text{则}\overrightarrow{X_1\left( t \right) Y_1\left( t \right) },\overrightarrow{X_2\left( t \right) Y_2\left( t \right) }\text{不相交}.\\
\end{cases}
\end{align}
将矩形$A_i(t)B_i(t)C_i(t)D_i(t)$和矩形$A_j(t)B_j(t)C_j(t)D_j(t)$的每一条边都代入上式,若所有边代入后都不相交,则这两个矩形无交点,否则有交点.

\subsection{判断板凳龙是否发生碰撞}\label{判断板凳龙是否发生碰撞}

在第$t$秒时,若对$\forall i,j\in\{1,2,\cdots,223\}$,将矩形$A_i(t)B_i(t)C_i(t)D_i(t)$和矩形$A_j(t)B_j(t)C_j(t)D_j(t)$代入\eqref{判断矩形相交的算法}后,结果都不相交.则板凳龙在第$t$秒不发生碰撞,否则发生碰撞.

\subsection{计算板凳龙发生碰撞的时间及此时板凳龙各把手的位置和速度}

令$t$依次取$1,2,\cdots$,代入\eqref{判断板凳龙是否发生碰撞},利用Python计算得到板凳龙第一次发生碰撞的时间$t_0$,从而板凳龙盘入的终止时刻为$t_0$.在将$t_0$代入\textbf{问题1}的模型,利用Python求解得到此时板凳龙各把手的位置直角坐标和速度见表.



\chapter{问题3}






\end{document}