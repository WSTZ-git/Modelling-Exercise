\documentclass[lang=cn,newtx,10pt,scheme=chinese]{../../Template/elegantbook}

\title{Intial Draft}
\subtitle{空}

\author{邹文杰}
\institute{无}
\date{2024/10/25}
\version{ElegantBook-4.5}
\bioinfo{自定义}{信息}

\extrainfo{宠辱不惊,闲看庭前花开花落;\\去留无意,漫随天外云卷云舒。}

\setcounter{tocdepth}{3}

\logo{logo-blue.png}
\cover{cover.png}

% 本文档额外使用的宏包和命令
\usepackage{../../Styles/mystyle-cumcmthesis}

\begin{document}

\maketitle

\frontmatter

\tableofcontents

\mainmatter

\chapter{问题一}

常物体、无内热源、稳态、一维导热问题。由于温区之间的间隙只考虑与之相邻的温区对其的影响,因此可以建立一维热传导方程:$\frac{\mathrm{d}^2T}{\mathrm{d}x^2}=0$(实际上,就是一维Laplace方程)。两次积分后解得:$T(x)=C_1x+C_2$。再将各个温区的初值条件代入即得:
\begin{align*}
T(x) &\equiv T_{1\sim 5}, &x &\in [0,5l_0 + 4d_0]\\
T(x) &= \frac{T_6 - T_{1\sim 5}}{d_0}x + T_{1\sim 5} - \frac{5l_0 + 4d_0}{d_0}(T_6 - T_{1\sim 5}), &x &\in [5l_0 + 4d_0,5l_0 + 5d_0]\\
T(x) &\equiv T_6, &x &\in [5l_0 + 5d_0,6l_0 + 5d_0]\\
T(x) &= \frac{T_7 - T_6}{d_0}x + T_6 - \frac{6l_0 + 5d_0}{d_0}(T_7 - T_6), &x &\in [6l_0 + 5d_0,6l_0 + 6d_0]\\
T(x) &\equiv T_7, &x &\in [6l_0 + 6d_0,7l_0 + 6d_0]\\
T(x) &= \frac{T_{8\sim 9} - T_7}{d_0}x + T_7 - \frac{7l_0 + 6d_0}{d_0}(T_{8\sim 9} - T_7), &x &\in [7l_0 + 6d_0,7l_0 + 7d_0]\\
T(x) &\equiv T_{8\sim 9}, &x &\in [7l_0 + 7d_0,9l_0 + 8d_0]\\
T(x) &= \frac{T_{10\sim 11} - T_{8\sim 9}}{d_0}x + T_{8\sim 9} - \frac{9l_0 + 8d_0}{d_0}(T_{10\sim 11} - T_{8\sim 9}), &x &\in [9l_0 + 8d_0,9l_0 + 9d_0]\\
T(x) &\equiv T_{10\sim 11}, &x &\in [9l_0 + 9d_0,11l_0 + 10d_0]
\end{align*}
由此可得传送带各位置达到稳态时的温度分布$T_{\infty}(x)=T_{\infty}(v_0t)$。

因为金属板与空气的传热(对流)系数$h$一般在$5\sim 500\mathrm{W}/(\mathrm{m}^2\cdot\mathrm{K})$内,金属的导热系数$\lambda$一般在$100\mathrm{W}/(\mathrm{m}^2\cdot\mathrm{K})$左右,而厚度$2\delta$很小,所以$Bi=\frac{\delta h}{\lambda}\ll 0.1$,因此可以用集中参数法(即可以将物体内部温度分布始终看作均匀)。

因为题目只给了金属板厚度$2\delta$,所以我们忽略水平方向的热对流,只考虑垂直于金属板方向的热对流对金属板温度的影响。根据对称性,我们只须考虑金属板上半部分的热对流即可。

垂直于金属板建立一维坐标系,垂直金属板向上为正方向.

建立PDE:
\begin{align*}
\frac{\partial T(x,t)}{\partial t}=a\frac{\partial ^2T(x,t)}{\partial x^2}(0<x<\delta,t>0)
\end{align*}
其中$a=\frac{\lambda}{\rho c}$.$\rho$为金属板密度,$c$为金属板热容.记$A=\rho c$,则$a=\frac{\lambda}{A}.$
边界(初值)条件(热传导方程第三边界条件):
\begin{align*}
&T(x,0)=T_0(=25^{\circ}\mathrm{C})(0\leqslant x\leqslant \delta)
\\
&\text{对称边界条件:}\frac{\partial T(x,t)}{\partial x}\Big|_{x = 0}=0
\\
&\text{对流边界条件:}h(T(\delta,t) - T_{\infty}(v_0t)) = -\lambda\frac{\partial T(x,t)}{\partial x}\Big|_{x = \delta}
\end{align*}
引入过余温度:
$\theta(x,t)=T(x,t)-T_{\infty}(v_0t)$
原方程可化为:
\begin{align*}
&\frac{\partial \theta(x,t)}{\partial t}=a\frac{\partial ^2\theta(x,t)}{\partial x^2}(0<x<\delta,t>0)\\
&\theta_0(t)=\theta(x,0)=T(x,0)-T_{\infty}(v_0t)=T_0-T_{\infty}(v_0t)\\
&\frac{\partial \theta(x,t)}{\partial x}\Big|_{x = 0}=0\\
&h\theta(\delta,t)=-\lambda\frac{\partial \theta(x,t)}{\partial x}\Big|_{x = \delta}
\end{align*}
于是利用PDE理论可得到解析解:
\begin{align}\label{1.1}
\frac{\theta(\eta,t)}{\theta_0(t)}=\sum_{n = 1}^{\infty}{C_n\exp(-\mu_{n}^{2}F_O)\cos(\mu_n\eta)}
\end{align}
其中各变量和参数如下:
\begin{align*}
&\theta(x,t)=T(x,t)-T_{\infty}(v_0t)\\
&\theta_0(t)=\theta(x,0)=T(x,0)-T_{\infty}(v_0t)=T_0-T_{\infty}(v_0t)\\
&C_n=\frac{2\sin(\mu_n)}{\mu_n+\cos(\mu_n)\sin(\mu_n)}\\
&\tan(\mu_n)=\frac{Bi}{\mu_n},n = 1,2,\cdots\\
&\eta=\frac{x}{\delta},F_O=\frac{at}{\delta ^2},Bi=\frac{h\delta}{\lambda}
\end{align*}

因此只需用最小二乘法估计参数$a,\lambda,h$即可.
\begin{align*}
\left( \widehat{A},\widehat{\lambda },\widehat{h} \right) =\underset{A,\lambda ,h}{\mathrm{arg}\min}\left( T\left( A,\lambda ,h,\delta,t \right) -T\left( \delta,t \right) \right) 
\end{align*}

假设空气是低速(静止)空气,则实际上,空气与金属板的对流系数$h$会只随着温差的改变而改变.
根据传热学公式可知,当我们只考虑热对流时,根据自然对流系数的经验公式就有
\begin{align*}
&h=\frac{\lambda \cdot Nu}{\delta}
\\
&Nu=B \cdot \left( Gr\cdot Pr \right) ^{m},B=1.076,m=\frac{1}{6}
\\
&Gr=\frac{g\left| T\left( \delta,t \right) -T_{\infty}\left( t \right) \right|\delta ^4}{T_fv^2}
\\
&T_f=\frac{T\left( \delta,t \right) +T_{\infty}\left( t \right)}{2}
\\
&10^4<Gr\cdot Pr<10^9
\end{align*}
其中$v,Pr$均为常数.从而$h$实际上是一个随时间变化的变量.将上式代入\eqref{1.1}式得
\begin{align*}
&\frac{T(\eta ,t)-T_{\infty}\left( v_0t \right)}{T_0-T_{\infty}\left( v_0t \right)}=\sum_{n=1}^{\infty}{C_n\exp\mathrm{(}-\mu _{n}^{2}F_O)\cos\mathrm{(}\mu _n\eta )}
\\
&\Rightarrow T(\eta ,t)=T_{\infty}\left( v_0t \right) +\left( T_0-T_{\infty}\left( v_0t \right) \right) \sum_{n=1}^{\infty}{C_n\exp\mathrm{(}-\mu _{n}^{2}F_O)\cos\mathrm{(}\mu _n\eta )}
\end{align*}
其中各变量和参数如下:
\begin{align*}
&C_n=\frac{2\sin\mathrm{(}\mu _n)}{\mu _n+\cos\mathrm{(}\mu _n)\sin\mathrm{(}\mu _n)}
\\
&\tan\mathrm{(}\mu _n)=\frac{Bi}{\mu _n},n=1,2,\cdots 
\\
&\eta =\frac{x}{\delta},F_O=\frac{at}{\delta ^2},Bi=\frac{h\delta}{\lambda}=Nu
\\
&Nu=B \cdot \left( Gr\cdot Pr \right) ^{m},B=1.076,m=\frac{1}{6}
\\
&Gr=\frac{g\left| T\left( \delta,t \right) -T_{\infty}\left( t \right) \right|\delta ^4}{T_fv^2}
\\
&T_f=\frac{T\left( \delta,t \right) +T_{\infty}\left( t \right)}{2}
\\
&10^4<Gr\cdot Pr<10^9
\end{align*}

于是我们根据题目给的一组数据,利用最小二乘法估计的参数应该为$a,\lambda,v,Pr$,故
\begin{align*}
&\left( \widehat{a},\widehat{\lambda },\widehat{v},\widehat{Pr} \right) =\underset{a,\lambda ,v,Pr}{\mathrm{arg}\min}\left( T\left( a,\lambda ,v,Pr,\delta,t \right) -T\left( \delta,t \right) \right) 
\\
&10^4<Gr\cdot Pr<10^9
\end{align*}
计算得到参数$a,\lambda,v,Pr$的最优值,即拟合优度$R^2$最接近1的取值.再将最优参数代入\eqref{1.1}得到炉温曲线.

后续三问就是分别做单目标优化而已,后续的模型建立不难,重点是模型求解的算法.















\end{document}